%\documentstyle[art10,titlepage,makeidx,twoside,EPSF/epsf,mytabbing]{j-article}

% euslisp
\newif\ifeuslisp
\euslisptrue

%%% added 2004.12.14
\documentclass[]{jarticle}
\usepackage{makeidx,mytabbing,fancyheadings}


\usepackage[dvipdfmx]{graphicx,color,epsfig}
\let\epsfile=\epsfig
\usepackage[dvipdfmx,bookmarks=true,bookmarksnumbered=true,bookmarkstype=toc]{hyperref}
\ifnum 42146=\euc"A4A2 \AtBeginDvi{\special{pdf:tounicode EUC-UCS2}}\else
\AtBeginDvi{\special{pdf:tounicode 90ms-RKSJ-UCS2}}\fi

%%%
\newcommand{\eusversion}{9.00}
\newcommand{\irteusversion}{1.00}


\flushbottom
\makeindex
\pagestyle{myheadings}
\oddsidemargin=0cm
\evensidemargin=0cm

% A4 size
\textwidth=16.5cm
\textheight=24.6cm
\topmargin=-0.8cm
\oddsidemargin= 0.5cm
\evensidemargin=0.5cm

% Letter size
%\topmargin=0.5cm
%\textwidth=17.6cm
%\textheight=23cm
%\oddsidemargin= 0.2cm
%\evensidemargin=0.2cm

\parindent=10pt
\parskip=1mm
%\baselineskip 14pt

\setcounter{totalnumber}{3}
\renewcommand{\topfraction}{0.99}       % 85% of a page (from page
                                        % top)
                                        % can be occupied by tbl / fig
                                        %
\renewcommand{\bottomfraction}{0.99}    % 85% of a page (from page
                                        % bottom)
                                        % can be occupied by tbl / fig
\renewcommand{\textfraction}{0.0}      % text shoud occupy 15% or
                                        % more in
                                        % a page
%\renewcommand{\floatpagefraction}{0.99} % 70% or more shoud occupy in
                                        % a
                                        % a float page


%%% removed 2004.12.14 \jintercharskip=0pt plus3.0pt minus1pt%

\usepackage{amsmath,amssymb}
\usepackage{arydshln}
\usepackage{mathrsfs}
\usepackage{cases} %% subnumcases
\usepackage{enumitem}

\newcommand{\labfig}[1]{\label{fig:#1}}
\newcommand{\labtab}[1]{\label{tab:#1}}
\newcommand{\labeq}[1]{\label{eq:#1}}
\newcommand{\labsec}[1]{\label{sec:#1}}
\newcommand{\labchap}[1]{\label{chap:#1}}
\newcommand{\labitem}[1]{\label{item:#1}}
\newcommand{\figlab}[1]{\labfig{#1}} % alias
\newcommand{\tablab}[1]{\labtab{#1}} % alias
\newcommand{\eqlab}[1]{\labeq{#1}} % alias
\newcommand{\eqlabel}[1]{\labeq{#1}} % alias
\newcommand{\equlab}[1]{\labeq{#1}} % alias
\newcommand{\reffig}[1]{{図~\ref{#1}}~}
\newcommand{\reftab}[1]{{Table~\ref{#1}}~}
\newcommand{\refeq}[1]{{式~(\ref{#1})}~}
\newcommand{\refchap}[1]{第\ref{#1}章}
\newcommand{\refitem}[1]{\ref{#1}}
\newcommand{\refsec}[1]{第\ref{#1}節}
\newcommand{\figref}[1]{\reffig{#1}} % alias
\newcommand{\tabref}[1]{\reftab{#1}} % alias
\renewcommand{\eqref}[1]{\refeq{#1}} % alias
\newcommand{\chapref}[1]{\refchap{#1}} % alias
\newcommand{\secref}[1]{\refsec{#1}} % alias
\newcommand{\bm}[1]{\mbox{\boldmath{$#1$}}}
\makeatletter
\newcommand{\footnoteref}[1]{\protected@xdef\@thefnmark{\ref{#1}}\@footnotemark}
\@addtoreset{equation}{section}
\def\theequation{\thesection.\arabic{equation}}
\makeatother
\newcommand{\eqdef}{\ensuremath{\stackrel{\mathrm{def}}{=}}}
\newcommand{\argmax}{\mathop{\rm arg~max}\limits}
\newcommand{\argmin}{\mathop{\rm arg~min}\limits}


\begin{document}

\newcommand{\ptext}[1]
{\tt \begin{quote} \begin{tabbing} #1 \end{tabbing} \end{quote} \rm}

\newcommand{\desclist}[1]{
\begin{list}{ }{\setlength{\rightmargin}{0mm}\topsep=0mm\partopsep=0mm}
\item #1
\end{list}
\vspace{3mm}}

\newcommand{\functiondescription}[4]{
\index{#1}
{\bf #1} \em #2 \rm \hfill [#3] 
%\if#4 \vspace{3mm} \\ \else \desclist{#4} \fi
%\ifx#4 \vspace{3mm} \\ \else \desclist{#4} \fi
 \desclist{\hspace{0mm}#4}
}

\newcommand{\bfx}[1]{\index{#1}{\bf #1}}
\newcommand{\emx}[1]{\index{#1}{\em #1}}

\newcommand{\longdescription}[3]{
\index{#1}
\begin{emtabbing}
{\bf #1} 
\it #2
\rm
\end{emtabbing}
\desclist{#3}
}

\newcommand{\funcdesc}[3]{\functiondescription{#1}{#2}{function}{#3}}
\newcommand{\macrodesc}[3]{\functiondescription{#1}{#2}{macro}{#3}}
\newcommand{\specialdesc}[3]{\functiondescription{#1}{#2}{special}{#3}}
\newcommand{\methoddesc}[3]{\functiondescription{#1}{#2}{method}{#3}}
\newcommand{\vardesc}[2]{\functiondescription{#1}{}{変数}{#2}}

\newcommand{\fundesc}[2]{\functiondescription{#1}{#2}{function}{\hspace{0mm}}}
\newcommand{\macdesc}[2]{\functiondescription{#1}{#2}{macro}{\hspace{0mm}}}
\newcommand{\spedesc}[2]{\functiondescription{#1}{#2}{special}{\hspace{0mm}}}
\newcommand{\metdesc}[2]{\functiondescription{#1}{#2}{method}{\hspace{0mm}}}

\newcommand{\constdesc}[2]{\functiondescription{#1}{}{定数}{#2}}

\newcommand{\classdesc}[4]{	%class, super slots description
\vspace{2mm} 
\index{#1}
{\Large {\bf #1 }} \hfill [class]  %super
\begin{tabbing}
\hspace{30mm} :super \hspace{5mm} \= {\bf #2} \\
\hspace{30mm} :slots \> #3 
\end{tabbing}
\vspace{4mm}
\desclist{#4}}

\newenvironment{refdesc}{
 \vspace{5mm} \parindent=0mm \topsep=0mm \parskip=0mm \leftmargin=10mm}{
             \parindent=10mm \topsep=3mm \parskip=1mm \leftmargin=0mm }


\date{}
\title{{\LARGE \bf 軌道最適化による動作生成 \\ リファレンスマニュアル} \\
\vspace{10mm}
{\large \today} \\
}

\author{
室岡雅樹 \\
murooka@jsk.t.u-tokyo.ac.jp \\
}

\thispagestyle{empty}
\maketitle
\pagenumbering{roman}
\tableofcontents

\newpage
\pagenumbering{arabic}

%%%%%%%%%%%%%%%%%%%%%%
\section{軌道最適化による動作生成の基礎} \label{chap:fundamental}
%%%%%
\subsection{タスク関数のノルムを最小にするコンフィギュレーションの探索}

$\bm{q} \in \mathbb{R}^{n_q}$を設計対象のコンフィギュレーションとする.
例えば一般の逆運動学計算では,
$\bm{q}$はある瞬間のロボットの関節角度を表すベクトルで,
コンフィギュレーションの次元$n_q$はロボットの関節自由度数となる.

動作生成問題を,
所望のタスクに対応するタスク関数$\bm{e}(\bm{q}): \mathbb{R}^{n_q} \to \mathbb{R}^{n_e}$について,
次式を満たす$\bm{q}$を得ることとして定義する.
\begin{eqnarray}
  \bm{e}(\bm{q}) = \bm{0} \label{eq:ik-eq}
\end{eqnarray}
例えば一般の逆運動学計算では,
$\bm{e}(\bm{q})$はエンドエフェクタの目標位置姿勢と現在位置姿勢の差を表す6次元ベクトルである.
非線形方程式(\ref{eq:ik-eq})の解を解析的に得ることは難しく,反復計算による数値解法が採られる.
\eqref{eq:ik-eq}が解をもたないときでも
最善のコンフィギュレーションを得られるように一般化すると,
\eqref{eq:ik-eq}の求解は次の最適化問題として表される
\footnote{
任意の半正定値行列$\bm{W}$に対して,
$\| \bm{e}(\bm{q}) \|_{\bm{W}}^2 = \bm{e}(\bm{q})^T\bm{W}\bm{e}(\bm{q}) = \bm{e}(\bm{q})^T\bm{S}^T\bm{S}\bm{e}(\bm{q}) = \| \bm{S} \bm{e}(\bm{q}) \|^2 $
を満たす$\bm{S}$が必ず存在するので,
\eqref{eq:ik-opt-1b}は任意の重み付きノルムを表現可能である.
}.
\begin{subequations}\label{eq:ik-opt-1}
\begin{eqnarray}
  &&\min_{\bm{q}} \ F(\bm{q}) \label{eq:ik-opt-1a} \\
  &&{\rm where} \ \ F(\bm{q}) \eqdef \frac{1}{2} \| \bm{e}(\bm{q}) \|^2 \label{eq:ik-opt-1b}
\end{eqnarray}
\end{subequations}
コンフィギュレーションが
最小値$\bm{q}_{\mathit{min}}$と最大値$\bm{q}_{\mathit{max}}$の間に含まれる必要があるとき,
逆運動学計算は次の制約付き非線形最適化問題として表される.
\begin{eqnarray}
  &&\min_{\bm{q}} \  F(\bm{q}) \hspace{4mm} {\rm s.t.} \ \  \bm{q}_{\mathit{min}} \leq \bm{q} \leq \bm{q}_{\mathit{max}} \label{eq:ik-opt-2}
\end{eqnarray}
例えば一般の逆運動学計算では,
$\bm{q}_{\mathit{min}}, \bm{q}_{\mathit{max}}$は関節角度の許容範囲の最小値,最大値を表す.
以降では,\eqref{eq:ik-opt-2}の制約を,より一般の形式である線形等式制約,線形不等式制約として次式のように表す
\footnote{
  \eqref{eq:ik-opt-2}における関節角度の最小値,最大値に関する制約は次式のように表される.
  \begin{eqnarray}
    && \bm{q}_{\mathit{min}} \leq \bm{q} \leq \bm{q}_{\mathit{max}} \nonumber \\
    \Leftrightarrow && \begin{pmatrix} \bm{I} \\ - \bm{I} \end{pmatrix} \bm{q} \geq \begin{pmatrix} \bm{q}_{\mathit{min}} \\ - \bm{q}_{\mathit{max}} \end{pmatrix} \nonumber
  \end{eqnarray}
}.
\begin{subequations}\label{eq:ik-opt-3}
\begin{eqnarray}
  &&\min_{\bm{q}} \  F(\bm{q}) \label{eq:ik-opt-3a} \\
  &&{\rm s.t.} \ \  \bm{A} \bm{q} = \bm{\bar{b}} \\
  &&\phantom{\rm s.t.} \ \  \bm{C} \bm{q} \geq \bm{\bar{d}}
\end{eqnarray}
\end{subequations}

制約付き非線形最適化問題の解法のひとつである逐次二次計画法では,
次の二次計画問題の最適解として得られる$\Delta \bm{q}_k^*$を用いて,$\bm{q}_{k+1} = \bm{q}_k + \Delta \bm{q}_k^*$として反復更新することで,
\eqref{eq:ik-opt-3}の最適解を導出する\footnote{\eqref{eq:sqp-1a}は$F(\bm{q})$を$\bm{q}_k$の周りでテーラー展開し三次以下の項を省略したものに一致する.
  逐次二次計画法については,以下の書籍の18章で詳しく説明されている.\\
  Numerical optimization,
  S. Wright and J. Nocedal,
  Springer Science,
  vol. 35,
  1999,
  \url{http://www.xn--vjq503akpco3w.top/literature/Nocedal_Wright_Numerical_optimization_v2.pdf}.
}.
\begin{subequations}\label{eq:sqp-1}
\begin{eqnarray}
  &&\min_{\Delta \bm{q}_k} \ F(\bm{q}_k) + \nabla F(\bm{q}_k)^T \Delta \bm{q}_k + \frac{1}{2} \Delta \bm{q}_k^T \nabla^2 F(\bm{q}_k) \Delta \bm{q}_k \label{eq:sqp-1a} \\
  &&{\rm s.t.} \ \ \bm{A} \Delta \bm{q}_k = \bm{\bar{b}} - \bm{A} \bm{q}_k \\
  &&\phantom{\rm s.t.} \ \ \bm{C} \Delta \bm{q}_k \geq \bm{\bar{d}} - \bm{C} \bm{q}_k
\end{eqnarray}
\end{subequations}
$\nabla F(\bm{q}_k), \nabla^2 F(\bm{q}_k)$はそれぞれ,
$F(\bm{q}_k)$の勾配,ヘッセ行列
\footnote{\eqref{eq:sqp-1a}の$\nabla^2 F(\bm{q}_k)$の部分は一般にはラグランジュ関数の$\bm{q}_k$に関するヘッセ行列であるが,等式・不等式制約が線形の場合は$F(\bm{q}_k)$のヘッセ行列と等価になる.}
で,
次式で表される.
\begin{subequations}
\begin{eqnarray}
  \nabla F(\bm{q}) &=& \left( \frac{\partial \bm{e}(\bm{q})}{\partial \bm{q}} \right)^T \bm{e}(\bm{q}) \label{eq:sqp-2a} \\
  &=& \bm{J}(\bm{q})^T \bm{e}(\bm{q}) \\
  \nabla^2 F(\bm{q}) &=& \sum_{i=1}^m e_i(\bm{q}) \nabla^2 e_i(\bm{q})
  + \left( \frac{\partial \bm{e}(\bm{q})}{\partial \bm{q}} \right)^T \frac{\partial \bm{e}(\bm{q})}{\partial \bm{q}} \label{eq:sqp-2b} \\
  &\approx& \left( \frac{\partial \bm{e}(\bm{q})}{\partial \bm{q}} \right)^T \frac{\partial \bm{e}(\bm{q})}{\partial \bm{q}} \label{eq:sqp-2c} \\
  &=& \bm{J}(\bm{q})^T \bm{J}(\bm{q})
\end{eqnarray}
\end{subequations}
ただし,
$e_i(\bm{q}) \ \ (i=1,2,\cdots,m)$は$\bm{e}(\bm{q})$の$i$番目の要素である.
\eqref{eq:sqp-2b}から\eqref{eq:sqp-2c}への変形では
$\bm{e}(\bm{q})$の二階微分がゼロであると近似している.
$\bm{J}(\bm{q}) \eqdef \frac{\partial \bm{e}(\bm{q})}{\partial \bm{q}} \in \mathbb{R}^{n_e \times n_q}$は$\bm{e}(\bm{q})$のヤコビ行列である.

\eqref{eq:sqp-2a}, \eqref{eq:sqp-2c}から
\eqref{eq:sqp-1a}の目的関数は次式で表される
\footnote{\eqref{eq:sqp-3b}は,以下の論文で紹介されている二次計画法によってコンフィギュレーション速度を導出する逆運動学解法における目的関数と一致する.\\
Feasible pattern generation method for humanoid robots,
F. Kanehiro et al.,
Proceedings of the 2009 IEEE-RAS International Conference on Humanoid Robots,
pp. 542-548,
2009.
}.
\begin{subequations}\label{eq:sqp-3}
\begin{eqnarray}
  &&\frac{1}{2} \bm{e}_k^T \bm{e}_k + \bm{e}_k^T \bm{J}_k \Delta \bm{q}_k + \frac{1}{2} \Delta \bm{q}_k^T \bm{J}_k^T \bm{J}_k \Delta \bm{q}_k \label{eq:sqp-3a} \\
  &=& \frac{1}{2} \| \bm{e}_k + \bm{J}_k \Delta \bm{q}_k \|^2 \label{eq:sqp-3b}
\end{eqnarray}
\end{subequations}
ただし,
$\bm{e}_k \eqdef \bm{e}(\bm{q}_k), \bm{J}_k \eqdef \bm{J}(\bm{q}_k)$とした.

結局,逐次二次計画法で反復的に解かれる二次計画問題(\ref{eq:sqp-1})は次式で表される.
\begin{subequations}\label{eq:sqp-4}
\begin{eqnarray}
  &&\min_{\Delta \bm{q}_k} \ \frac{1}{2} \Delta \bm{q}_k^T \bm{J}_k^T \bm{J}_k \Delta \bm{q}_k + \bm{e}_k^T \bm{J}_k \Delta \bm{q}_k \label{eq:sqp-4a} \\
  &&{\rm s.t.} \ \ \bm{A} \Delta \bm{q}_k = \bm{b} \\
  &&\phantom{\rm s.t.} \ \ \bm{C} \Delta \bm{q}_k \geq \bm{d}
\end{eqnarray}
\end{subequations}
ここで,
\begin{eqnarray}
  \bm{b} &=& \bm{\bar{b}} - \bm{A} \bm{q}_k \\
  \bm{d} &=& \bm{\bar{d}} - \bm{C} \bm{q}_k
\end{eqnarray}
とおいた.

%%%%%
\subsection{コンフィギュレーション二次形式の正則化項の追加}

\eqref{eq:ik-opt-1a}の最適化問題の目的関数を,次式の$\hat{F}(\bm{q})$で置き換える.
\begin{eqnarray}
  && \hat{F}(\bm{q}) = F(\bm{q}) + F_{\mathit{reg}}(\bm{q}) \\
  && {\rm where} \ \ F_{\mathit{reg}}(\bm{q}) = \frac{1}{2} \bm{q}^T \bm{\bar{W}}_{\mathit{reg}} \bm{q}
\end{eqnarray}

目的関数$\hat{F}(\bm{q})$の勾配,ヘッセ行列は次式で表される.
\begin{subequations}
\begin{eqnarray}
  \nabla \hat{F}(\bm{q})
  &=& \nabla F(\bm{q}) + \nabla F_{\mathit{reg}}(\bm{q}) \\
  &=& \bm{J}(\bm{q})^T \bm{e}(\bm{q}) + \bm{\bar{W}}_{\mathit{reg}} \bm{q} \\
  \nabla^2 \hat{F}(\bm{q}) &=& \nabla^2 F(\bm{q}) + \nabla^2 F_{\mathit{reg}}(\bm{q}) \\
  &\approx& \bm{J}(\bm{q})^T \bm{J}(\bm{q}) + \bm{\bar{W}}_{\mathit{reg}}
\end{eqnarray}
\end{subequations}

したがって,\eqref{eq:sqp-4}の二次計画問題は次式で表される.
\begin{subequations}\label{eq:sqp-5}
\begin{eqnarray}
  &&\min_{\Delta \bm{q}_k} \ \frac{1}{2} \Delta \bm{q}_k^T \left( \bm{J}_k^T \bm{J}_k + \bm{\bar{W}}_{\mathit{reg}} \right) \Delta \bm{q}_k + \left( \bm{J}_k^T \bm{e}_k + \bm{\bar{W}}_{\mathit{reg}} \bm{q}_k \right)^T \Delta \bm{q}_k \label{eq:sqp-5a} \\
  &&{\rm s.t.} \ \ \bm{A} \Delta \bm{q}_k = \bm{b} \\
  &&\phantom{\rm s.t.} \ \ \bm{C} \Delta \bm{q}_k \geq \bm{d}
\end{eqnarray}
\end{subequations}

%%%%%
\subsection{コンフィギュレーション更新量の正則項の追加}

Gauss-Newton法とLevenberg-Marquardt法の比較を参考に,
\eqref{eq:sqp-5a}の二次形式項の行列に,次式のように微小な係数をかけた単位行列を加えると,一部の適用例について逐次二次計画法の収束性が改善された
\footnote{これは,最適化における信頼領域(trust region)に関連している.}.
\begin{subequations}\label{eq:sqp-6}
\begin{eqnarray}
  &&\min_{\Delta \bm{q}_k} \ \frac{1}{2} \Delta \bm{q}_k^T \left( \bm{J}_k^T \bm{J}_k + \bm{\bar{W}}_{\mathit{reg}} + \lambda \bm{I} \right) \Delta \bm{q}_k + \left( \bm{J}_k^T \bm{e}_k + \bm{\bar{W}}_{\mathit{reg}} \bm{q}_k \right)^T \Delta \bm{q}_k \label{eq:sqp-6a} \\
  &&{\rm s.t.} \ \ \bm{A} \Delta \bm{q}_k = \bm{b} \\
  &&\phantom{\rm s.t.} \ \ \bm{C} \Delta \bm{q}_k \geq \bm{d}
\end{eqnarray}
\end{subequations}
改良誤差減衰最小二乗法
\footnote{
Levenberg-Marquardt法による可解性を問わない逆運動学,
杉原 知道,
日本ロボット学会誌,
vol. 29,
no. 3,
pp. 269-277,
2011.
}
を参考にすると,
$\lambda$は次式のように決定される.
\begin{eqnarray}
  \lambda = \lambda_r F(\bm{q}_k) + w_r
\end{eqnarray}
$\lambda_r$と$w_r$は正の定数である.

%%%%%
\subsection{ソースコードと数式の対応}

\begin{subequations}
\begin{eqnarray}
  \bm{W}_{\mathit{reg}} &\eqdef& \bm{\bar{W}}_{\mathit{reg}} + \lambda \bm{I} \\
  \bm{v}_{\mathit{reg}} &\eqdef& \bm{\bar{W}}_{\mathit{reg}} \bm{q}_k
\end{eqnarray}
\end{subequations}
とすると,\eqref{eq:sqp-6}は次式で表される.
\begin{subequations}\label{eq:sqp-7}
\begin{eqnarray}
  &&\min_{\Delta \bm{q}_k} \ \frac{1}{2} \Delta \bm{q}_k^T \left( \bm{J}_k^T \bm{J}_k + \bm{W} \right) \Delta \bm{q}_k + \left( \bm{J}_k^T \bm{e}_k + \bm{v}_{\mathit{reg}} \right)^T \Delta \bm{q}_k \label{eq:sqp-7a} \\
  &&{\rm s.t.} \ \ \bm{A} \Delta \bm{q}_k = \bm{b} \\
  &&\phantom{\rm s.t.} \ \ \bm{C} \Delta \bm{q}_k \geq \bm{d}
\end{eqnarray}
\end{subequations}

\secref{chap:config-task}や\chapref{chap:extended}で説明する
{\it ***-configuration-task}クラスのメソッドは
\eqref{eq:sqp-7}中の記号と以下のように対応している.

\begin{description}[labelindent=10mm, labelwidth=70mm]
  \setlength{\itemsep}{-2pt}
  \item[{\it :config-vector}] get $\bm{q}$
  \item[{\it :set-config}] set $\bm{q}$
  \item[{\it :task-value}] get $\bm{e}(\bm{q})$
  \item[{\it :task-jacobian}] get $\bm{J}(\bm{q}) \eqdef \frac{\partial \bm{e}(\bm{q})}{\partial \bm{q}}$
  \item[{\it :config-equality-constraint-matrix}] get $\bm{A}$
  \item[{\it :config-equality-constraint-vector}] get $\bm{b}$
  \item[{\it :config-inequality-constraint-matrix}] get $\bm{C}$
  \item[{\it :config-inequality-constraint-vector}] get $\bm{d}$
  \item[{\it :regular-matrix}] get $\bm{W}_{\mathit{reg}}$
  \item[{\it :regular-vector}] get $\bm{v}_{\mathit{reg}}$
\end{description}

%%%%%
\subsection{章の構成}

\chapref{chap:config-task}では,
コンフィギュレーション$\bm{q}$の取得・更新,タスク関数$\bm{e}(\bm{q})$の取得,タスク関数のヤコビ行列$\bm{J}(\bm{q}) \eqdef \frac{\partial \bm{e}(\bm{q})}{\partial \bm{q}}$の取得,コンフィギュレーションの等式・不等式制約$\bm{A}, \bm{b}, \bm{C}, \bm{d}$の取得のためのクラスを説明する.
\secref{sec:instant-config-task}ではコンフィギュレーション$\bm{q}$が瞬時の情報,
\secref{sec:trajectory-config-task}ではコンフィギュレーション$\bm{q}$が時系列の情報を表す場合をそれぞれ説明する.

\chapref{chap:sqp}では,\chapref{chap:config-task}で説明されるクラスを用いて逐次二次計画法により最適化を行うためのクラスを説明する.

\chapref{chap:extended}では,
用途に応じて拡張されたコンフィギュレーションとタスク関数のクラスを説明する.
\secref{sec:manip}では,
マニピュレーションのために,ロボットに加えて物体のコンフィギュレーションを計画する場合を説明する.
\secref{sec:bspline}では,
ロボットの関節位置の軌道をBスプライン関数でパラメトリックに表現する場合を説明する.
いずれにおいても,最適化では\chapref{chap:sqp}で説明された逐次二次計画法のクラスが利用される.

\chapref{chap:appendix}では,その他の補足事項を説明する.
\secref{sec:base-extention}では,jskeusで定義されているクラスの拡張について説明する.
\secref{sec:robot-environment}では,環境との接触を有するロボットの問題設定を記述するためのクラスについて説明する.
\secref{sec:torque-jacobian}では,関節トルクを関節角度で微分したヤコビ行列を導出するための関数について説明する.


%%%%%%%%%%%%%%%%%%%%%%
\section{コンフィギュレーションとタスク関数} \label{chap:config-task}
%%%%%
\subsection{瞬時コンフィギュレーションと瞬時タスク関数} \label{sec:instant-config-task}
\input{instant-configuration-task}
%%%%%
\subsection{軌道コンフィギュレーションと軌道タスク関数} \label{sec:trajectory-config-task}
\input{trajectory-configuration-task}

%%%%%%%%%%%%%%%%%%%%%%
\section{勾配を用いた制約付き非線形最適化} \label{chap:sqp}
%%%%%
\subsection{逐次二次計画法} \label{sec:sqp}
\input{sqp-optimization}
%%%%%
\subsection{複数解候補を用いた逐次二次計画法} \label{sec:sqp-msc}
\subsubsection{複数解候補を用いた逐次二次計画法の理論}
\eqref{eq:ik-opt-3a}の最適化問題に逐次二次計画法などの制約付き非線形最適化手法を適用すると,
初期値から勾配方向に進行して至る局所最適解が得られると考えられる.
したがって解は初期値に強く依存する.

\eqref{eq:ik-opt-3a}の代わりに,以下の最適化問題を考える.
\begin{eqnarray}
  &&\min_{\bm{\hat{q}}} \ \sum_{i \in \mathcal{I}} \left\{ F(\bm{q}^{(i)}) + k_{\mathit{msc}} F_{\mathit{msc}}(\bm{\hat{q}}; i) \right\} \label{eq:solution-candidate-opt} \\
  &&{\rm s.t.} \ \  \bm{A} \bm{q}^{(i)} = \bm{\bar{b}} \ \ \ \ i \in \mathcal{I} \\
  &&\phantom{\rm s.t.} \ \  \bm{C} \bm{q}^{(i)} \geq \bm{\bar{d}} \ \ \ \ i \in \mathcal{I} \\
  &&{\rm where} \ \ \bm{\hat{q}} \eqdef \begin{pmatrix} \bm{q}^{(1)T} & \bm{q}^{(2)T} & \cdots & \bm{q}^{(N_{\mathit{msc}})T} \end{pmatrix}^T \\
  &&\phantom{\rm where} \ \ \mathcal{I} \eqdef \{ 1,2,\cdots,N_{\mathit{msc}} \} \\
  &&\phantom{\rm where} \ \ F_{\mathit{msc}}(\bm{\hat{q}}; i) \eqdef - \frac{1}{2} \sum_{\substack{j \in \mathcal{I} \\ j \not= i}} \log \| \bm{d}(\bm{q}^{(i)}, \bm{q}^{(j)}) \|^2 \\
  &&\phantom{\rm where} \ \ \bm{d}(\bm{q}^{(i)}, \bm{q}^{(j)}) \eqdef \bm{q}^{(i)} - \bm{q}^{(j)}
\end{eqnarray}
$N_{\mathit{msc}}$は解候補の個数で,事前に与えるものとする.$\mathit{msc}$は複数解候補(multiple solution candidates)を表す.
これは,複数の解候補を同時に探索し,それぞれの解候補$\bm{q}^{(i)}$が本来の目的関数$F(\bm{q}^{(i)})$を小さくして,なおかつ,解候補どうしの距離が大きくなるように最適化することを表している.
これにより,初期値に依存した唯一の局所解だけでなく,そこから離れた複数の局所解を得ることが可能となり,通常の最適化に比べてより良い解が得られることが期待される.
以降では,解候補どうしの距離のコストを表す項$F_{\mathit{msc}}(\bm{\hat{q}}; i)$を解候補分散項と呼ぶ
\footnote{解分散項の$\log$を無くすことは適切ではない.なぜなら,$d = \| \bm{d}(\bm{q}^{(i)}, \bm{q}^{(j)}) \|$として,解分散項の勾配は,
\begin{eqnarray}
  \frac{\partial}{\partial d}\left(- \frac{1}{2} \log d^2 \right) = - \frac{1}{d} \to - \infty \ \ (d \to +0) \hspace{10mm}
  \frac{\partial}{\partial d}\left(- \frac{1}{2} \log d^2 \right) = - \frac{1}{d} \to 0 \ \ (d \to \infty)
\end{eqnarray}
となり,最適化により,コンフィギュレーションが近いときほど離れるように更新し,遠くなるとその影響が小さくなる効果が期待される.それに対し,$\log$がない場合の勾配は,
\begin{eqnarray}
  \frac{\partial}{\partial d}\left(- \frac{1}{2} d^2 \right) = - d \to 0 \ \ (d \to +0) \hspace{10mm}
  \frac{\partial}{\partial d}\left(- \frac{1}{2} d^2 \right) = - d \to - \infty \ \ (d \to \infty)
\end{eqnarray}
となり,コンフィギュレーションが遠くなるほど離れるように更新し,近いときはその影響が小さくなる.これは,コンフィギュレーションが一致する勾配ゼロの点と,無限に離れ発散する最適値をもち,これらは最適化において望まない挙動をもたらす.
}.

解候補分散項のヤコビ行列,ヘッセ行列の各成分は次式で得られる\footnote{ヘッセ行列の導出は以下を参考にした.\url{https://math.stackexchange.com/questions/175263/gradient-and-hessian-of-general-2-norm}}.
\begin{subequations}
\begin{eqnarray}
  \nabla_i F_{\mathit{msc}}(\bm{\hat{q}}; i) &=& \frac{\partial F_{\mathit{msc}}(\bm{\hat{q}}; i)}{\partial \bm{q}^{(i)}} \\
  &=& - \frac{1}{2} \sum_{\substack{j \in \mathcal{I} \\ j \not= i}} \frac{\partial}{\partial \bm{q}^{(i)}} \log \| \bm{d}(\bm{q}^{(i)}, \bm{q}^{(j)}) \|^2 \\
  &=& - \sum_{\substack{j \in \mathcal{I} \\ j \not= i}} \frac{1}{\| \bm{d}(\bm{q}^{(i)}, \bm{q}^{(j)}) \|^2} \left( \frac{\partial \bm{d}(\bm{q}^{(i)}, \bm{q}^{(j)})}{\partial \bm{q}^{(i)}} \right)^T \bm{d}(\bm{q}^{(i)}, \bm{q}^{(j)})\\
  &=& - \sum_{\substack{j \in \mathcal{I} \\ j \not= i}} \frac{\bm{d}(\bm{q}^{(i)}, \bm{q}^{(j)})}{\| \bm{d}(\bm{q}^{(i)}, \bm{q}^{(j)}) \|^2} \\
\end{eqnarray}
\end{subequations}
\begin{subequations}
\begin{eqnarray}
  \nabla_k F_{\mathit{msc}}(\bm{\hat{q}}; i) &=& \frac{\partial F_{\mathit{msc}}(\bm{\hat{q}}; i)}{\partial \bm{q}^{(k)}} \ \ \ \ k \in \mathcal{I} \land k \not= i \\
  &=& - \frac{1}{2} \sum_{\substack{j \in \mathcal{I} \\ j \not= i}} \frac{\partial}{\partial \bm{q}^{(k)}} \log \| \bm{d}(\bm{q}^{(i)}, \bm{q}^{(j)}) \|^2 \\
  &=& - \frac{1}{2} \frac{\partial}{\partial \bm{q}^{(k)}} \log \| \bm{d}(\bm{q}^{(i)}, \bm{q}^{(k)}) \|^2 \\
  &=& - \frac{1}{\| \bm{d}(\bm{q}^{(i)}, \bm{q}^{(k)}) \|^2} \left( \frac{\partial \bm{d}(\bm{q}^{(i)}, \bm{q}^{(k)})}{\partial \bm{q}^{(k)}} \right)^T \bm{d}(\bm{q}^{(i)}, \bm{q}^{(k)})\\
  &=& \frac{\bm{d}(\bm{q}^{(i)}, \bm{q}^{(k)})}{\| \bm{d}(\bm{q}^{(i)}, \bm{q}^{(k)}) \|^2} \\
\end{eqnarray}
\end{subequations}
\begin{subequations}
\begin{eqnarray}
  \nabla^2_{ii} F_{\mathit{msc}}(\bm{\hat{q}}; i) &=& \frac{\partial^2 F_{\mathit{msc}}(\bm{\hat{q}}; i)}{\partial \bm{q}^{(i) 2}} \\
  &=& - \sum_{\substack{j \in \mathcal{I} \\ j \not= i}} \frac{\partial}{\partial \bm{q}^{(i)}} \left( \left\{ \| \bm{d}(\bm{q}^{(i)}, \bm{q}^{(j)}) \|^2 \right\}^{-1} \bm{d}(\bm{q}^{(i)}, \bm{q}^{(j)}) \right) \\
  &=& - \sum_{\substack{j \in \mathcal{I} \\ j \not= i}} \left( - 2 \left\{ \| \bm{d}(\bm{q}^{(i)}, \bm{q}^{(j)}) \|^2 \right\}^{-2} \bm{d}(\bm{q}^{(i)}, \bm{q}^{(j)}) \bm{d}(\bm{q}^{(i)}, \bm{q}^{(j)})^T + \left\{ \| \bm{d}(\bm{q}^{(i)}, \bm{q}^{(j)}) \|^2 \right\}^{-1} \bm{I} \right) \\
  &=& - \sum_{\substack{j \in \mathcal{I} \\ j \not= i}} \left( - \frac{2}{\| \bm{d}(\bm{q}^{(i)}, \bm{q}^{(j)}) \|^4} \bm{d}(\bm{q}^{(i)}, \bm{q}^{(j)}) \bm{d}(\bm{q}^{(i)}, \bm{q}^{(j)})^T + \frac{1}{ \| \bm{d}(\bm{q}^{(i)}, \bm{q}^{(j)}) \|^2 } \bm{I} \right) \\
  &=& - \sum_{\substack{j \in \mathcal{I} \\ j \not= i}} \bm{H}(\bm{q}^{(i)}, \bm{q}^{(j)})
\end{eqnarray}
\end{subequations}
ただし,
\begin{eqnarray}
  \bm{H}(\bm{q}^{(i)}, \bm{q}^{(j)}) \eqdef
  - \frac{2}{\| \bm{d}(\bm{q}^{(i)}, \bm{q}^{(j)}) \|^4} \bm{d}(\bm{q}^{(i)}, \bm{q}^{(j)}) \bm{d}(\bm{q}^{(i)}, \bm{q}^{(j)})^T + \frac{1}{ \| \bm{d}(\bm{q}^{(i)}, \bm{q}^{(j)}) \|^2 } \bm{I}
\end{eqnarray}
\begin{subequations}
\begin{eqnarray}
  \nabla^2_{ik} F_{\mathit{msc}}(\bm{\hat{q}}; i) &=& \frac{\partial^2 F_{\mathit{msc}}(\bm{\hat{q}}; i)}{\partial \bm{q}^{(i)} \partial \bm{q}^{(k)}} \ \ \ \ k \in \mathcal{I} \land k \not= i \\
  &=& - \sum_{\substack{j \in \mathcal{I} \\ j \not= i}} \frac{\partial}{\partial \bm{q}^{(k)}} \left( \left\{ \| \bm{d}(\bm{q}^{(i)}, \bm{q}^{(j)}) \|^2 \right\}^{-1} \bm{d}(\bm{q}^{(i)}, \bm{q}^{(j)}) \right) \\
  &=& - \frac{\partial}{\partial \bm{q}^{(k)}} \left( \left\{ \| \bm{d}(\bm{q}^{(i)}, \bm{q}^{(k)}) \|^2 \right\}^{-1} \bm{d}(\bm{q}^{(i)}, \bm{q}^{(k)}) \right) \\
  &=& - \left( 2 \left\{ \| \bm{d}(\bm{q}^{(i)}, \bm{q}^{(k)}) \|^2 \right\}^{-2} \bm{d}(\bm{q}^{(i)}, \bm{q}^{(k)}) \bm{d}(\bm{q}^{(i)}, \bm{q}^{(k)})^T - \left\{ \| \bm{d}(\bm{q}^{(i)}, \bm{q}^{(k)}) \|^2 \right\}^{-1} \bm{I} \right) \\
  &=& - \frac{2}{\| \bm{d}(\bm{q}^{(i)}, \bm{q}^{(k)}) \|^4} \bm{d}(\bm{q}^{(i)}, \bm{q}^{(k)}) \bm{d}(\bm{q}^{(i)}, \bm{q}^{(k)})^T + \frac{1}{ \| \bm{d}(\bm{q}^{(i)}, \bm{q}^{(k)}) \|^2 } \bm{I} \\
  &=& \bm{H}(\bm{q}^{(i)}, \bm{q}^{(k)})
\end{eqnarray}
\end{subequations}
\begin{subequations}
\begin{eqnarray}
  \nabla^2_{kk} F_{\mathit{msc}}(\bm{\hat{q}}; i) &=& \frac{\partial^2 F_{\mathit{msc}}(\bm{\hat{q}}; i)}{\partial \bm{q}^{(k)2}} \ \ \ \ k \in \mathcal{I} \land k \not= i\\
  &=& \frac{\partial}{\partial \bm{q}^{(k)}} \left( \left\{ \| \bm{d}(\bm{q}^{(i)}, \bm{q}^{(k)}) \|^2 \right\}^{-1} \bm{d}(\bm{q}^{(i)}, \bm{q}^{(k)}) \right) \\
  &=& - \left( - \frac{2}{\| \bm{d}(\bm{q}^{(i)}, \bm{q}^{(k)}) \|^4} \bm{d}(\bm{q}^{(i)}, \bm{q}^{(k)}) \bm{d}(\bm{q}^{(i)}, \bm{q}^{(k)})^T + \frac{1}{ \| \bm{d}(\bm{q}^{(i)}, \bm{q}^{(k)}) \|^2 } \bm{I} \right) \\
  &=& - \bm{H}(\bm{q}^{(i)}, \bm{q}^{(k)})
\end{eqnarray}
\end{subequations}
\begin{subequations}
\begin{eqnarray}
  \nabla^2_{kl} F_{\mathit{msc}}(\bm{\hat{q}}; i) &=& \frac{\partial^2 F_{\mathit{msc}}(\bm{\hat{q}}; i)}{\partial \bm{q}^{(k)} \partial \bm{q}^{(l)}} \ \ \ \ k \in \mathcal{I} \land l \in \mathcal{I} \land k \not= i \land l \not= i \land k \not= l\\
  &=& \frac{\partial}{\partial \bm{q}^{(l)}} \left( \left\{ \| \bm{d}(\bm{q}^{(i)}, \bm{q}^{(k)}) \|^2 \right\}^{-1} \bm{d}(\bm{q}^{(i)}, \bm{q}^{(k)}) \right) \\
  &=& \bm{O}
\end{eqnarray}
\end{subequations}
したがって,
解候補分散項のヤコビ行列,ヘッセ行列は次式で表される.
\begin{subequations}
\begin{eqnarray}
  \nabla F_{\mathit{msc}}(\bm{\hat{q}}; i) &=& \frac{\partial F_{\mathit{msc}}(\bm{\hat{q}}; i)}{\partial \bm{\hat{q}}} \\
  &=&
  \begin{pmatrix}
  \frac{\bm{d}(\bm{q}^{(i)}, \bm{q}^{(1)})}{\| \bm{d}(\bm{q}^{(i)}, \bm{q}^{(1)}) \|^2} \\
  \vdots \\
  \frac{\bm{d}(\bm{q}^{(i)}, \bm{q}^{(i-1)})}{\| \bm{d}(\bm{q}^{(i)}, \bm{q}^{(i-1)}) \|^2} \\
  - \sum_{\substack{j \in \mathcal{I} \\ j \not= i}} \frac{\bm{d}(\bm{q}^{(i)}, \bm{q}^{(j)})}{\| \bm{d}(\bm{q}^{(i)}, \bm{q}^{(j)}) \|^2} \\
  \frac{\bm{d}(\bm{q}^{(i)}, \bm{q}^{(i+1)})}{\| \bm{d}(\bm{q}^{(i)}, \bm{q}^{(i+1)}) \|^2} \\
  \vdots \\
  \frac{\bm{d}(\bm{q}^{(i)}, \bm{q}^{(N_{\mathit{msc}})})}{\| \bm{d}(\bm{q}^{(i)}, \bm{q}^{(N_{\mathit{msc}})}) \|^2}
  \end{pmatrix} \\
  \bm{v}_{\mathit{msc}} &\eqdef&
  \sum_{i \in \mathcal{I}} \nabla F_{\mathit{msc}}(\bm{\hat{q}}; i) \\
  &=&
  2
  \begin{pmatrix}
  - \sum_{\substack{j \in \mathcal{I} \\ j \not= 1}} \frac{\bm{d}(\bm{q}^{(1)}, \bm{q}^{(j)})}{\| \bm{d}(\bm{q}^{(1)}, \bm{q}^{(j)}) \|^2} \\
  \vdots \\
  - \sum_{\substack{j \in \mathcal{I} \\ j \not= N_{\mathit{msc}}}} \frac{\bm{d}(\bm{q}^{(N_{\mathit{msc}})}, \bm{q}^{(j)})}{\| \bm{d}(\bm{q}^{(N_{\mathit{msc}})}, \bm{q}^{(j)}) \|^2} \\
  \end{pmatrix} \label{eq:sqp-msc-dispersion-matrix}
\end{eqnarray}
\end{subequations}
\begin{subequations}
\begin{eqnarray}
  \nabla^2 F_{\mathit{msc}}(\bm{\hat{q}}; i) &=& \frac{\partial^2 F_{\mathit{msc}}(\bm{\hat{q}}; i)}{\partial \bm{\hat{q}}^2} \\
  &=&
  \bordermatrix{
    & 1 & \cdots & i-1 & i & i+1 & \cdots & N_{\mathit{msc}} \cr
    1 & - \bm{H}_{i,1} & & & \bm{H}_{i,1} & & &\cr
    \vdots & & \ddots & & \vdots & & & \cr
    i-1 & & & - \bm{H}_{i,i-1} & \bm{H}_{i,i-1} & & & \cr
    i & \bm{H}_{i,1} & \cdots & \bm{H}_{i,i-1} & - \sum_{\substack{j \in \mathcal{I} \\ j \not= i}} \bm{H}_{i,j} & \bm{H}_{i,i+1} & \cdots & \bm{H}_{i,N_{\mathit{msc}}} \cr
    i+1 & & & & \bm{H}_{i,i+1} & - \bm{H}_{i,i+1} & & \cr
    \vdots & & & & \vdots & & \ddots & \cr
    N_{\mathit{msc}} & & & & \bm{H}_{i,N_{\mathit{msc}}} & & & - \bm{H}_{i, N_{\mathit{msc}}} \cr
  } \\
  \bm{W}_{\mathit{msc}} &\eqdef&
  \sum_{i \in \mathcal{I}} \nabla^2 F_{\mathit{msc}}(\bm{\hat{q}}; i) \\
  &=&
  2
  \begin{pmatrix}
    - \sum_{\substack{j \in \mathcal{I} \\ j \not= i}} \bm{H}_{1,j} & \bm{H}_{1,2} & \cdots & \bm{H}_{1,N_{\mathit{msc}}} \\
    \bm{H}_{2,1} & - \sum_{\substack{j \in \mathcal{I} \\ j \not= i}} \bm{H}_{2,j} & & \bm{H}_{2,N_{\mathit{msc}}} \\
    \vdots & & \ddots & \vdots \\
    \bm{H}_{N_{\mathit{msc}},1} & \bm{H}_{N_{\mathit{msc}},2} & \cdots & - \sum_{\substack{j \in \mathcal{I} \\ j \not= i}} \bm{H}_{N_{\mathit{msc}},j}
  \end{pmatrix} \label{eq:sqp-msc-dispersion-vector}
\end{eqnarray}
\end{subequations}
ただし,$\bm{H}(\bm{q}^{(i)}, \bm{q}^{(j)})$を$\bm{H}_{i,j}$と略して記す.
また,$\bm{d}(\bm{q}^{(i)}, \bm{q}^{(j)}) = - \bm{d}(\bm{q}^{(j)}, \bm{q}^{(i)}), \ \bm{H}_{i,j} = \bm{H}_{j,i}$を利用した.

解候補分散項$\sum_{i \in \mathcal{I}} F_{\mathit{msc}}(\bm{\hat{q}}; i)$による二次計画問題の目的関数(\eqref{eq:sqp-1a})は次式で表される.
\begin{eqnarray}
  &&\sum_{i \in \mathcal{I}} \left\{ F_{\mathit{msc}}(\bm{\hat{q}}_k; i) + \nabla F_{\mathit{msc}}(\bm{\hat{q}}_k; i)^T \Delta \bm{\hat{q}}_k + \frac{1}{2} \Delta \bm{\hat{q}}_k^T \nabla^2 F_{\mathit{msc}}(\bm{\hat{q}}_k; i) \Delta \bm{\hat{q}}_k \right\} \\
  &=&
  \sum_{i \in \mathcal{I}} F_{\mathit{msc}}(\bm{\hat{q}}_k; i)
  + \left\{ \sum_{i \in \mathcal{I}} \nabla F_{\mathit{msc}}(\bm{\hat{q}}_k; i) \right\}^T \Delta \bm{\hat{q}}_k
  + \frac{1}{2} \Delta \bm{\hat{q}}_k^T \left\{ \sum_{i \in \mathcal{I}} \nabla^2 F_{\mathit{msc}}(\bm{\hat{q}}_k; i) \right\} \Delta \bm{\hat{q}}_k \\
  &=&
  \sum_{i \in \mathcal{I}} F_{\mathit{msc}}(\bm{\hat{q}}_k; i) + \bm{v}_{\mathit{msc}}^T \Delta \bm{\hat{q}}_k + \frac{1}{2} \Delta \bm{\hat{q}}_k^T \bm{W}_{\mathit{msc}} \Delta \bm{\hat{q}}_k
\end{eqnarray}

$\bm{W}_{\mathit{msc}}$が必ずしも半正定値行列ではないことに注意する必要がある.
以下のようにして$\bm{W}_{\mathit{msc}}$に近い正定値行列を計算し用いることで対処する
\footnote{$\bm{W}_{\mathit{msc}}$が対称行列であることから,以下を参考にした.\url{https://math.stackexchange.com/questions/648809/how-to-find-closest-positive-definite-matrix-of-non-symmetric-matrix\#comment1689831_649522}}.
$\bm{W}_{\mathit{msc}}$が次式のように固有値分解されるとする.
\begin{eqnarray}
  \bm{W}_{\mathit{msc}} = \bm{V}_{\mathit{msc}} \bm{D}_{\mathit{msc}} \bm{V}_{\mathit{msc}}^{-1}
\end{eqnarray}
ただし,$\bm{D}_{\mathit{msc}}$は固有値を対角成分にもつ対角行列,$\bm{V}_{\mathit{msc}}$は固有ベクトルを並べた行列である.
このとき$\bm{W}_{\mathit{msc}}$に近い正定値行列$\bm{\tilde{W}}_{\mathit{msc}}$は次式で得られる.
\begin{eqnarray}
  \bm{\tilde{W}}_{\mathit{msc}} = \bm{V}_{\mathit{msc}} \bm{D}_{\mathit{msc}}^+ \bm{V}_{\mathit{msc}}^{-1}
\end{eqnarray}
ただし,$\bm{D}_{\mathit{msc}}^+$は$\bm{D}_{\mathit{msc}}$の対角成分のうち,負のものを$0$で置き換えた対角行列である.

\eqref{eq:solution-candidate-opt}において,
解候補を分散させながら,最終的に本来の目的関数を最小にする解を得るために,
SQPのイテレーションごとに,解候補分散項のスケール$k_{\mathit{msc}}$を次式のように更新することが有効である.
\begin{eqnarray}
  k_{\mathit{msc}} \gets \min ( \gamma_{\mathit{msc}} k_{\mathit{msc}}, k_{\mathit{msc\mathchar`-min}})
\end{eqnarray}
$\gamma_{\mathit{msc}}$は$0< \gamma_{\mathit{msc}} < 1$なるスケール減少率,
$k_{\mathit{msc\mathchar`-min}}$はスケール最小値を表す.

\subsubsection{複数解候補を用いた逐次二次計画法の実装}
\input{sqp-msc-optimization}

%%%%%%%%%%%%%%%%%%%%%%
\section{動作生成の拡張} \label{chap:extended}
%%%%%
\subsection{マニピュレーションの動作生成} \label{sec:manip}
\input{robot-object-environment}
\input{instant-manipulation-configuration-task}
%%%%%
\subsection{Bスプラインを用いた関節軌道生成} \label{sec:bspline}
\subsubsection{Bスプラインを用いた関節軌道生成の理論}
%%
\subsubsection*{一般のBスプライン基底関数の定義}

Bスプライン基底関数は以下で定義される.
\begin{eqnarray}
  b_{i, 0}(t) &\eqdef& \left\{ \begin{array}{ll} 1 & {\rm if \ \ } t_i \leq t < t_{i+1}\\ 0 & {\rm otherwise}\end{array}\right. \label{eq:bspline-def-1} \\
  b_{i, n}(t) &\eqdef& \frac{t - t_i}{t_{i+n} - t_i} b_{i, n-1}(t) + \frac{t_{i+n+1} - t}{t_{i+n+1} - t_{i+1}} b_{i+1, n-1}(t) \label{eq:bspline-def-2}
\end{eqnarray}
$t_i$はノットと呼ばれる.

%%
\subsubsection*{使用区間を指定してノットを一様とする場合のBスプライン基底関数}

$t_s, t_f$をBスプラインの使用区間の初期,終端時刻とする.

$n < m$とする.
\begin{eqnarray}
  t_n = t_s \\
  t_m = t_f
\end{eqnarray}
とする.$t_i \ (0 \leq i \leq n+m)$が等間隔に並ぶとすると,
\begin{eqnarray}
  t_i &=& \frac{i - n}{m - n} (t_f - t_s) + t_s \\
  &=& h i + \frac{m t_s - n t_f}{m - n} \label{eq:bspline-t-def-1}
\end{eqnarray}
ただし,
\begin{eqnarray}
  h \eqdef \frac{t_f - t_s}{m - n}
\end{eqnarray}
\eqref{eq:bspline-t-def-1}を\eqref{eq:bspline-def-1}, \eqref{eq:bspline-def-2}に代入すると,
Bスプライン基底関数は次式で得られる.
\begin{eqnarray}
  b_{i, 0}(t) &=& \left\{ \begin{array}{ll} 1 & {\rm if \ \ } t_i \leq t < t_{i+1}\\ 0 & {\rm otherwise}\end{array}\right. \label{eq:bspline-uniform-knot-1} \\
  b_{i, n}(t) &=& \frac{(t - t_i) b_{i, n-1}(t) + (t_{i+n+1} - t) b_{i+1, n-1}(t)}{n h} \label{eq:bspline-uniform-knot-2}
\end{eqnarray}
以降では,$n$をBスプラインの次数,$m$を制御点の個数と呼ぶ.

%%
\subsubsection*{Bスプラインの凸包性}

\eqref{eq:bspline-uniform-knot-1},\eqref{eq:bspline-uniform-knot-2}で定義されるBスプライン基底関数$b_{i,n}(t)$は次式のように凸包性を持つ.
\begin{eqnarray}
  &&\sum_{i=0}^{m-1} b_{i,n}(t) = 1 \ \ (t_s \leq t \leq t_f) \label{eq:bspline-convex-1} \\
  &&0 \leq b_{i,n}(t) \leq 1 \ \ (i = 0,1,\cdots,m-1, \ t_s \leq t \leq t_f) \label{eq:bspline-convex-2}
\end{eqnarray}

%%
\subsubsection*{Bスプラインの微分}

Bスプライン基底関数の微分に関して次式が成り立つ\footnote{数学的帰納法で証明できる.\url{http://mat.fsv.cvut.cz/gcg/sbornik/prochazkova.pdf}}.
\begin{eqnarray}
  &&\bm{\dot{b}}_n(t) = \frac{d \bm{b}_n(t)}{d t} = \bm{D} \bm{b}_{n-1}(t) \label{eq:bspline-derivative}
\end{eqnarray}
ただし,
\begin{eqnarray}
  \bm{b}_n(t) &\eqdef& \begin{pmatrix} b_{0,n}(t) \\ b_{1,n}(t) \\ \vdots \\ b_{m-1,n}(t) \end{pmatrix} \in \mathbb{R}^m \\
  \bm{D} &\eqdef& \frac{1}{h} \begin{pmatrix} 1 & -1 &&&\bm{O}\\ & 1 & -1 &&\\&&\ddots&\ddots&\\&&&\ddots&-1\\\bm{O}&&&&1\end{pmatrix} \in \mathbb{R}^{m \times m}
\end{eqnarray}
したがって,$k$階微分に関して次式が成り立つ.
\begin{eqnarray}
  &&\bm{b}_n^{(k)}(t) = \frac{d^{(k)} \bm{b}_n(t)}{d t^{(k)}} = \bm{D}^k \bm{b}_{n-k}(t) \label{eq:bspline-derivative-k}
\end{eqnarray}

%%
\subsubsection*{Bスプラインによる関節角軌道の表現}

$j$番目の関節角軌道$\theta_j (t)$を次式で表す.
\begin{eqnarray}
  \theta_j (t) \eqdef \sum_{i=0}^{m-1} p_{j,i} b_{i, n}(t) = \bm{p}_j^T \bm{b}_n(t) \in \mathbb{R} \ \ (t_s \leq t \leq t_f) \label{eq:bspline-theta-j}
\end{eqnarray}
ただし,
\begin{eqnarray}
  \bm{p}_j = \begin{pmatrix} p_{j,0} \\ p_{j,1} \\ \vdots \\ p_{j,m-1} \end{pmatrix} \in \mathbb{R}^m, \ \ 
  \bm{b}_n(t) = \begin{pmatrix} b_{0,n}(t) \\ b_{1,n}(t) \\ \vdots \\ b_{m-1,n}(t) \end{pmatrix} \in \mathbb{R}^m
\end{eqnarray}
以降では,$\bm{p}_j$を制御点,$\bm{b}_n(t)$を基底関数と呼ぶ.
制御点$\bm{p}_j$を決定すると関節角軌道が定まる.制御点$\bm{p}_j$を動作計画の設計変数とする.

$j=1,2,\cdots,N_{\mathit{joint}}$番目の関節角軌道を並べたベクトル関数は,
\begin{eqnarray}
  \bm{\theta}(t) \eqdef \begin{pmatrix} \theta_1(t) \\ \theta_2(t) \\ \vdots \\ \theta_{N_{\mathit{joint}}}(t) \end{pmatrix}
  = \begin{pmatrix} \bm{p}_1^T \bm{b}_n(t) \\ \bm{p}_2^T \bm{b}_n(t) \\ \vdots \\ \bm{p}_{N_{\mathit{joint}}}^T \bm{b}_n(t) \end{pmatrix}
  = \begin{pmatrix} \bm{p}_1^T \\ \bm{p}_2^T \\ \vdots \\ \bm{p}_{N_{\mathit{joint}}}^T \end{pmatrix} \bm{b}_n(t)
  = \bm{P} \bm{b}_n(t)
  \in \mathbb{R}^{N_{\mathit{joint}}} \label{eq:spline-theta-vec}
\end{eqnarray}
ただし,
\begin{eqnarray}
  \bm{P} \eqdef \begin{pmatrix} \bm{p}_1^T \\ \bm{p}_2^T \\ \vdots \\ \bm{p}_{N_{\mathit{joint}}}^T \end{pmatrix} \in \mathbb{R}^{N_{\mathit{joint}} \times m}
\end{eqnarray}

\eqref{eq:spline-theta-vec}は,制御点を縦に並べたベクトルとして分離して,以下のようにも表現できる.
\begin{eqnarray}
  \bm{\theta}(t) = \begin{pmatrix} \theta_1(t) \\ \theta_2(t) \\ \vdots \\ \theta_{N_{\mathit{joint}}}(t) \end{pmatrix}
  = \begin{pmatrix} \bm{b}_n^T(t) \bm{p}_1 \\  \bm{b}_n^T(t) \bm{p}_2 \\  \vdots \\ \bm{b}_n^T(t) \bm{p}_{N_{\mathit{joint}}} \end{pmatrix}
  = \begin{pmatrix} \bm{b}_n^T(t)&&&\bm{O}\\&\bm{b}_n^T(t)&&\\&&\ddots&\\\bm{O}&&&\bm{b}_n^T(t) \end{pmatrix}
  \begin{pmatrix} \bm{p}_1 \\  \bm{p}_2 \\  \vdots \\ \bm{p}_{N_{\mathit{joint}}} \end{pmatrix}
  = \bm{B}_n(t) \bm{p}
  \in \mathbb{R}^{N_{\mathit{joint}}} \label{eq:spline-theta-vec-2}
\end{eqnarray}
ただし,
\begin{eqnarray}
  \bm{B}_n(t) \eqdef \begin{pmatrix} \bm{b}_n^T(t)&&&\bm{O}\\&\bm{b}_n^T(t)&&\\&&\ddots&\\\bm{O}&&&\bm{b}_n^T(t) \end{pmatrix} \in \mathbb{R}^{N_{\mathit{joint}} \times m N_{\mathit{joint}}}, \ \
  \bm{p} \eqdef \begin{pmatrix} \bm{p}_1 \\  \bm{p}_2 \\  \vdots \\ \bm{p}_{N_{\mathit{joint}}} \end{pmatrix} \in \mathbb{R}^{m N_{\mathit{joint}}}
\end{eqnarray}

%%
\subsubsection*{Bスプラインによる関節角軌道の微分}

\eqref{eq:spline-theta-vec}と\eqref{eq:bspline-derivative}から,関節角速度軌道は次式で得られる.
\begin{eqnarray}
  \bm{\dot{\theta}}(t) &=& \bm{P} \bm{\dot{b}}_n(t) \\
  &=& \bm{P} \bm{D} \bm{b}_{n-1}(t) \\
  &=& \begin{pmatrix} \bm{p}_1^T \\ \vdots \\ \bm{p}_{N_{\mathit{joint}}}^T \end{pmatrix} \bm{D} \bm{b}_{n-1}(t) \\
  &=& \begin{pmatrix} \bm{p}_1^T \bm{D} \bm{b}_{n-1}(t) \\ \vdots \\ \bm{p}_{N_{\mathit{joint}}}^T \bm{D} \bm{b}_{n-1}(t) \end{pmatrix} \\
  &=& \begin{pmatrix} \bm{b}_{n-1}^T(t) \bm{D}^T \bm{p}_1 \\ \vdots \\ \bm{b}_{n-1}^T(t) \bm{D}^T \bm{p}_{N_{\mathit{joint}}} \end{pmatrix} \\
  &=& \begin{pmatrix} \bm{b}_{n-1}^T(t) \bm{D}^T && \bm{O} \\ &\ddots& \\ \bm{O} && \bm{b}_{n-1}^T(t) \bm{D}^T \end{pmatrix} \begin{pmatrix} \bm{p}_1 \\ \vdots \\ \bm{p}_{N_{\mathit{joint}}} \end{pmatrix} \\
  &=& \begin{pmatrix} \bm{b}_{n-1}^T(t) && \bm{O} \\ &\ddots& \\ \bm{O} && \bm{b}_{n-1}^T(t) \end{pmatrix} \begin{pmatrix} \bm{D}^T && \bm{O} \\ &\ddots& \\ \bm{O} && \bm{D}^T \end{pmatrix} \begin{pmatrix} \bm{p}_1 \\ \vdots \\ \bm{p}_{N_{\mathit{joint}}} \end{pmatrix} \\
  &=& \bm{B}_{n-1}(t) \bm{\hat{D}}_1 \bm{p}
\end{eqnarray}
ただし,
\begin{eqnarray}
  \bm{\hat{D}}_1 &=& \begin{pmatrix} \bm{D}^T&&&\bm{O}\\&\bm{D}^T&&\\&&\ddots&\\\bm{O}&&&\bm{D}^T \end{pmatrix} \in \mathbb{R}^{m N_{\mathit{joint}} \times m N_{\mathit{joint}}}
\end{eqnarray}
同様にして,関節角軌道の$k$階微分は次式で得られる.
\begin{eqnarray}
  \bm{\theta}^{(k)}(t) &=& \frac{d^{(k)} \bm{\theta}(t)}{d t^{(k)}} \\
  &=& \bm{P} \bm{D}^k \bm{b}_{n-k}(t) \label{eq:spline-theta-dot-k-1} \\
  &=& \bm{B}_{n-k}(t) \bm{\hat{D}}_k \bm{p} \label{eq:spline-theta-dot-k-2}
\end{eqnarray}
ただし,
\begin{eqnarray}
  \bm{\hat{D}}_k &=& \begin{pmatrix} (\bm{D}^k)^T&&\bm{O}\\&\ddots&\\\bm{O}&&(\bm{D}^k)^T \end{pmatrix} = (\bm{\hat{D}}_1)^k \in \mathbb{R}^{m N_{\mathit{joint}} \times m N_{\mathit{joint}}}
\end{eqnarray}
計算時間は\eqref{eq:spline-theta-dot-k-1}のほうが\eqref{eq:spline-theta-dot-k-2}より速い.

%%
\subsubsection*{エンドエフェクタ位置姿勢拘束のタスク関数}

関節角$\bm{\theta} \in \mathbb{R}^{N_{\mathit{joint}}}$からエンドエフェクタ位置姿勢$\bm{r} \in \mathbb{R}^6$への写像を$\bm{f}(\bm{\theta})$で表す.
\begin{eqnarray}
  \bm{r} = \bm{f}(\bm{\theta})
\end{eqnarray}

関節角軌道が\eqref{eq:spline-theta-vec-2}で表現されるとき,エンドエフェクタ軌道は次式で表される.
\begin{eqnarray}
  \bm{r}(t) = \bm{f}(\bm{\theta}(t)) = \bm{f}(\bm{B}_n(t) \bm{p})
\end{eqnarray}

$l = 1,2,\cdots,{N_{\mathit{tm}}}$について,時刻$t_l$でエンドエフェクタの位置姿勢が$\bm{r}_l$であるタスクのタスク関数は次式で表される.
以降では,$t_l$をタイミングと呼ぶ.
\begin{eqnarray}
  \bm{e}(\bm{p}, \bm{t}) \eqdef
  \begin{pmatrix} \bm{e}_1(\bm{p}, \bm{t}) \\ \bm{e}_2(\bm{p}, \bm{t}) \\ \vdots \\ \bm{e}_{N_{\mathit{tm}}}(\bm{p}, \bm{t}) \end{pmatrix} =
  \begin{pmatrix} \bm{r}_1 - \bm{f}(\bm{\theta}(t_1)) \\ \bm{r}_2 - \bm{f}(\bm{\theta}(t_2)) \\ \vdots \\ \bm{r}_{N_{\mathit{tm}}} - \bm{f}(\bm{\theta}(t_{N_{\mathit{tm}}})) \end{pmatrix} =
  \begin{pmatrix} \bm{r}_1 - \bm{f}(\bm{B}_n(t_1)\bm{p}) \\ \bm{r}_2 - \bm{f}(\bm{B}_n(t_2)\bm{p}) \\ \vdots \\ \bm{r}_{N_{\mathit{tm}}} - \bm{f}(\bm{B}_n(t_{N_{\mathit{tm}}})\bm{p}) \end{pmatrix} \in \mathbb{R}^{6{N_{\mathit{tm}}}} \label{eq:bspline-task}
\end{eqnarray}
ただし,
\begin{eqnarray}
  \bm{e}_l(\bm{p}, \bm{t}) &\eqdef& \bm{r}_l - \bm{f}(\bm{\theta}(t_l)) = \bm{r}_l - \bm{f}(\bm{B}_n(t_l)\bm{p}) \in \mathbb{R}^6 \ (l = 1,2,\cdots,{N_{\mathit{tm}}}) \\
  \bm{t} &\eqdef& \begin{pmatrix} t_1 \\ t_2 \\ \vdots \\ t_{N_{\mathit{tm}}} \end{pmatrix} \in \mathbb{R}^{N_{\mathit{tm}}}
\end{eqnarray}

このタスクを実現する関節角軌道は,次の評価関数を最小にする制御点$\bm{p}$,タイミング$\bm{t}$を求めることで導出することができる.
\begin{eqnarray}
  F(\bm{p}, \bm{t}) &\eqdef& \frac{1}{2} \| \bm{e}(\bm{p}, \bm{t}) \|^2 \\
  &=& \frac{1}{2} \sum_{l=1}^{{N_{\mathit{tm}}}} \| \bm{r}_l - \bm{f}(\bm{\theta}(t_l)) \|^2 \\
  &=& \frac{1}{2} \sum_{l=1}^{{N_{\mathit{tm}}}} \| \bm{r}_l - \bm{f}(\bm{B}_n(t_l) \bm{p}) \|^2 \label{eq:bspline-objective}
\end{eqnarray}

また,$l$番目の幾何拘束の許容誤差を$\bm{e}_{\mathit{tol},l} \geq \bm{0} \in \mathbb{R}^6$とする場合,タスク関数$\bm{\tilde{e}}_{l}(\bm{p}, \bm{t})$は次式で表される.
\begin{eqnarray}
  \tilde{e}_{l,i}(\bm{p}, \bm{t}) \eqdef
  \left\{ \begin{array}{ll}
    e_{l,i}(\bm{p}, \bm{t}) - e_{\mathit{tol},l,i} & e_{l,i}(\bm{p}, \bm{t}) > e_{\mathit{tol},l,i} \\
    e_{l,i}(\bm{p}, \bm{t}) + e_{\mathit{tol},l,i} & e_{l,i}(\bm{p}, \bm{t}) < - e_{\mathit{tol},l,i} \\
    0 & {\rm otherwise} \\
  \end{array} \right. \ \ (i = 1,2,\cdots,6)
\end{eqnarray}
$\tilde{e}_{l,i}(\bm{p}, \bm{t})$は$\bm{\tilde{e}}_{l}(\bm{p}, \bm{t})$の$i$番目の要素である.
$e_{l,i}(\bm{p}, \bm{t})$は$\bm{e}(\bm{p}, \bm{t})$の$i$番目の要素である.

%%
\subsubsection*{タスク関数を制御点で微分したヤコビ行列}

\eqref{eq:bspline-objective}を目的関数とする最適化問題をGauss-Newton法,Levenberg-Marquardt法や逐次二次計画法で解く場合,
タスク関数(\ref{eq:bspline-task})のヤコビ行列が必要となる.

各時刻でのエンドエフェクタ位置姿勢拘束のタスク関数$\bm{e}_l(\bm{p}, \bm{t})$の制御点$\bm{p}$に対するヤコビ行列は次式で求められる.
\begin{eqnarray}
  \frac{\partial \bm{e}_l(\bm{p}, \bm{t})}{\partial \bm{p}} &=& \frac{\partial}{\partial \bm{p}} \{ \bm{r}_l - \bm{f}(\bm{B}_n(t_l)\bm{p}) \} \\
  &=& - \frac{\partial}{\partial \bm{p}} \bm{f}(\bm{B}_n(t_l)\bm{p}) \\
  &=& - \left. \frac{\partial \bm{f}}{\partial \bm{\theta}} \right|_{\bm{\theta} = \bm{\theta}(t_l)} \frac{\partial \bm{\theta}}{\partial \bm{p}} \\
  &=& - \bm{J}(\bm{\theta}(t_l)) \frac{\partial}{\partial \bm{p}} \{ \bm{B}_n(t_l)\bm{p} \} \\
  &=& - \bm{J}(\bm{\theta}(t_l)) \bm{B}_n(t_l) \label{eq:bspline-task-jacobian-with-control}
\end{eqnarray}
途中の変形で,$\bm{\theta}(\bm{p}; t) = \bm{B}_n(t) \bm{p}$であることを利用した.
ただし,
\begin{eqnarray}
  \bm{J} \eqdef \frac{\partial \bm{f}}{\partial \bm{\theta}}
\end{eqnarray}

%%
\subsubsection*{タスク関数をタイミングで微分したヤコビ行列}

各時刻でのエンドエフェクタ位置姿勢拘束のタスク関数$\bm{e}_l(\bm{p}, \bm{t})$のタイミング$\bm{t}$に対するヤコビ行列は次式で求められる.
\begin{eqnarray}
  \frac{\partial \bm{e}_l(\bm{p}, \bm{t})}{\partial t_l} &=& \frac{\partial}{\partial t_l} \{ \bm{r}_l - \bm{f}(\bm{P}\bm{b}_n(t_l)) \} \\
  &=& - \frac{\partial}{\partial t_l} \bm{f}(\bm{P}\bm{b}_n(t_l)) \\
  &=& - \left. \frac{\partial \bm{f}}{\partial \bm{\theta}} \right|_{\bm{\theta} = \bm{\theta}(t_l)} \frac{\partial \bm{\theta}}{\partial t_l} \\
  &=& - \bm{J}(\bm{\theta}(t_l)) \frac{\partial}{\partial t_l} \{ \bm{P}\bm{b}_n(t_l) \} \\
  &=& - \bm{J}(\bm{\theta}(t_l)) \bm{P} \bm{\dot{b}}_n(t_l) \\
  &=& - \bm{J}(\bm{\theta}(t_l)) \bm{P} \bm{D} \bm{b}_{n-1}(t_l) \label{eq:bspline-task-jacobian-with-timing}
\end{eqnarray}
途中の変形で,$\bm{\theta}(\bm{p}; t) = \bm{P} \bm{b}_n(t)$であることを利用した.

%%
\subsubsection*{初期・終端関節速度・加速度のタスク関数とヤコビ行列}

初期,終端時刻の関節速度,加速度はゼロであるべきである.
タスク関数は次式となる.
\begin{eqnarray}
  \bm{e}_{sv}(\bm{p}, \bm{t})
  &\eqdef& \bm{\dot{\theta}}(t_s) \\
  &=& \bm{B}_{n-1}(t_s) \bm{\hat{D}}_1 \bm{p} \\
  &=& \bm{P} \bm{D} \bm{b}_{n-1}(t_s) \\
  \bm{e}_{fv}(\bm{p}, \bm{t})
  &\eqdef& \bm{\dot{\theta}}(t_f) \\
  &=& \bm{B}_{n-1}(t_f) \bm{\hat{D}}_1 \bm{p} \\
  &=& \bm{P} \bm{D} \bm{b}_{n-1}(t_f) \\
  \bm{e}_{sa}(\bm{p}, \bm{t})
  &\eqdef& \bm{\ddot{\theta}}(t_s) \\
  &=& \bm{B}_{n-2}(t_s) \bm{\hat{D}}_2 \bm{p} \\
  &=& \bm{P} \bm{D}^2 \bm{b}_{n-2}(t_s) \\
  \bm{e}_{fa}(\bm{p}, \bm{t})
  &\eqdef& \bm{\ddot{\theta}}(t_f) \\
  &=& \bm{B}_{n-2}(t_f) \bm{\hat{D}}_2 \bm{p} \\
  &=& \bm{P} \bm{D}^2 \bm{b}_{n-2}(t_f)
\end{eqnarray}

制御点で微分したヤコビ行列は次式で表される.
\begin{eqnarray}
  \frac{\partial \bm{e}_{sv}(\bm{p}, \bm{t})}{\partial \bm{p}} &=& \bm{B}_{n-1}(t_s) \bm{\hat{D}}_1 \label{eq:bspline-stationery-task-jacobian-with-control-sv} \\
  \frac{\partial \bm{e}_{fv}(\bm{p}, \bm{t})}{\partial \bm{p}} &=& \bm{B}_{n-1}(t_f) \bm{\hat{D}}_1 \label{eq:bspline-stationery-task-jacobian-with-control-fv} \\
  \frac{\partial \bm{e}_{sa}(\bm{p}, \bm{t})}{\partial \bm{p}} &=& \bm{B}_{n-2}(t_s) \bm{\hat{D}}_2 \label{eq:bspline-stationery-task-jacobian-with-control-sa} \\
  \frac{\partial \bm{e}_{fa}(\bm{p}, \bm{t})}{\partial \bm{p}} &=& \bm{B}_{n-2}(t_f) \bm{\hat{D}}_2 \label{eq:bspline-stationery-task-jacobian-with-control-fa}
\end{eqnarray}

初期時刻,終端時刻で微分したヤコビ行列は次式で表される.
\begin{eqnarray}
  \frac{\partial \bm{e}_{sv}(\bm{p}, \bm{t})}{\partial t_s} &=& \bm{P} \bm{D} \frac{\partial \bm{b}_{n-1}(t_s)}{\partial t_s}
  = \bm{P} \bm{D}^2 \bm{b}_{n-2}(t_s) \label{eq:bspline-stationery-task-jacobian-with-timing-sv} \\
  \frac{\partial \bm{e}_{fv}(\bm{p}, \bm{t})}{\partial t_f} &=& \bm{P} \bm{D} \frac{\partial \bm{b}_{n-1}(t_f)}{\partial t_f}
  = \bm{P} \bm{D}^2 \bm{b}_{n-2}(t_f) \label{eq:bspline-stationery-task-jacobian-with-timing-fv} \\
  \frac{\partial \bm{e}_{sa}(\bm{p}, \bm{t})}{\partial t_s} &=& \bm{P} \bm{D}^2 \frac{\partial \bm{b}_{n-2}(t_s)}{\partial t_s}
  = \bm{P} \bm{D}^3 \bm{b}_{n-3}(t_s) \label{eq:bspline-stationery-task-jacobian-with-timing-sa} \\
  \frac{\partial \bm{e}_{fa}(\bm{p}, \bm{t})}{\partial t_f} &=& \bm{P} \bm{D}^2 \frac{\partial \bm{b}_{n-2}(t_f)}{\partial t_f}
  = \bm{P} \bm{D}^3 \bm{b}_{n-3}(t_f) \label{eq:bspline-stationery-task-jacobian-with-timing-fa}
\end{eqnarray}

%%
\subsubsection*{関節角上下限制約}

\eqref{eq:bspline-theta-j}の関節角軌道定義において,
\begin{eqnarray}
  \bm{p}_j \leq \theta_{max,j} \bm{1}_m
\end{eqnarray}
のとき,Bスプラインの凸包性(\eqref{eq:bspline-convex-1}, \eqref{eq:bspline-convex-2})より次式が成り立つ.
ただし,$\bm{1}_m \in \mathbb{R}^m$は全要素が$1$の$m$次元ベクトルである.
\begin{eqnarray}
  \theta_j (t) &=& \sum_{i=0}^{m-1} p_{j,i} b_{i, n}(t) \\
  &\leq& \sum_{i=0}^{m-1} \theta_{max,j} b_{i, n}(t) \\
  &=& \theta_{max,j} \sum_{i=0}^{m-1} b_{i, n}(t) \\
  &=& \theta_{max,j}
\end{eqnarray}
同様に,$\theta_{min,j} \bm{1}_m \leq \bm{p}_j$とすれば,$\theta_{min,j} \leq \theta_j (t)$が成り立つ.

したがって,$j$番目の関節角の上下限を$\theta_{max,j}, \theta_{min,j}$とすると,次式の制約を制御点に課すことで,関節角上下限制約を満たす関節角軌道が得られる.
\begin{eqnarray}
  \theta_{min,j} \bm{1}_m \leq \bm{p}_j \leq \theta_{max,j} \bm{1}_m \ (j = 1,2,\cdots,N_{\mathit{joint}})
\end{eqnarray}
つまり,
\begin{eqnarray}
  &&\bm{\hat{E}} \bm{\theta}_{min} \leq \bm{p} \leq \bm{\hat{E}} \bm{\theta}_{max} \label{eq:bspline-theta-constraint} \\
  \Leftrightarrow&&
  \begin{pmatrix} \bm{I} \\ - \bm{I} \end{pmatrix} \bm{p} \geq \begin{pmatrix} \bm{\hat{E}} \bm{\theta}_{min} \\ - \bm{\hat{E}} \bm{\theta}_{max} \end{pmatrix}
\end{eqnarray}
ただし,
\begin{eqnarray}
  \bm{\hat{E}} \eqdef \begin{pmatrix} \bm{1}_m&&&\bm{0}_m\\&\bm{1}_m&&\\&&\ddots&\\\bm{0}_m&&&\bm{1}_m \end{pmatrix} \in \mathbb{R}^{m N_{\mathit{joint}} \times N_{\mathit{joint}}}
\end{eqnarray}
これは,逐次二次計画法の中で,次式の不等式制約となる.
\begin{eqnarray}
  \begin{pmatrix} \bm{I} \\ - \bm{I} \end{pmatrix} \Delta \bm{p} \geq \begin{pmatrix} \bm{\hat{E}} \bm{\theta}_{min} - \bm{p} \\ - \bm{\hat{E}} \bm{\theta}_{max} + \bm{p} \end{pmatrix}
\end{eqnarray}

%%
\subsubsection*{関節角速度・角加速度上下限制約}

\eqref{eq:bspline-theta-j}と\eqref{eq:bspline-derivative}より,関節角速度軌道,角加速度軌道は次式で表される.
\begin{eqnarray}
  \dot{\theta}_j (t) &=& \bm{p}_j^T \bm{\dot{b}}_n(t) = \bm{p}_j^T \bm{D} \bm{b}_{n-1}(t) = (\bm{D}^T \bm{p}_j)^T \bm{b}_{n-1}(t) \in \mathbb{R} \ \ (t_s \leq t \leq t_f) \label{eq:bspline-vel-j} \\
  \ddot{\theta}_j (t) &=& \bm{p}_j^T \bm{\ddot{b}}_n(t) = \bm{p}_j^T \bm{D}^2 \bm{b}_{n-2}(t) = ((\bm{D}^2)^T \bm{p}_j)^T \bm{b}_{n-2}(t) \in \mathbb{R} \ \ (t_s \leq t \leq t_f) \label{eq:bspline-acc-j}
\end{eqnarray}

$j$番目の関節角速度,角加速度の上限を$v_{max,j}, a_{max,j}$とする.
関節角上下限制約の導出と同様に考えると,
次式の制約を制御点に課すことで,関節角速度・角加速度上下限制約を満たす関節角軌道が得られる.
\begin{eqnarray}
  &&- v_{max,j} \bm{1}_m \leq \bm{D}^T \bm{p}_j \leq v_{max,j} \bm{1}_m \ (j = 1,2,\cdots,N_{\mathit{joint}}) \\
  &&- a_{max,j} \bm{1}_m \leq (\bm{D}^2)^T \bm{p}_j \leq a_{max,j} \bm{1}_m \ (j = 1,2,\cdots,N_{\mathit{joint}})
\end{eqnarray}
つまり,
\begin{eqnarray}
  &&- \bm{\hat{E}} \bm{v}_{max} \leq \bm{\hat{D}}_1 \bm{p} \leq \bm{\hat{E}} \bm{v}_{max} \label{eq:bspline-theta-dot-constraint} \\
  \Leftrightarrow&&
  \begin{pmatrix} \bm{\hat{D}}_1 \\ - \bm{\hat{D}}_1 \end{pmatrix} \bm{p} \geq \begin{pmatrix} - \bm{\hat{E}} \bm{v}_{max} \\ - \bm{\hat{E}} \bm{v}_{max} \end{pmatrix} \\
  &&- \bm{\hat{E}} \bm{a}_{max} \leq \bm{\hat{D}}_2 \bm{p} \leq \bm{\hat{E}} \bm{a}_{max} \label{eq:bspline-theta-ddot-constraint} \\
  \Leftrightarrow&&
  \begin{pmatrix} \bm{\hat{D}}_2 \\ - \bm{\hat{D}}_2 \end{pmatrix} \bm{p} \geq \begin{pmatrix} - \bm{\hat{E}} \bm{a}_{max} \\ - \bm{\hat{E}} \bm{a}_{max} \end{pmatrix}
\end{eqnarray}
これは,逐次二次計画法の中で,次式の不等式制約となる.
\begin{eqnarray}
  &&\begin{pmatrix} \bm{\hat{D}}_1 \\ - \bm{\hat{D}}_1 \end{pmatrix} \Delta \bm{p} \geq \begin{pmatrix} - \bm{\hat{E}} \bm{v}_{max} - \bm{\hat{D}}_1 \bm{p} \\ - \bm{\hat{E}} \bm{v}_{max} + \bm{\hat{D}}_1 \bm{p} \end{pmatrix} \\
  &&\begin{pmatrix} \bm{\hat{D}}_2 \\ - \bm{\hat{D}}_2 \end{pmatrix} \Delta \bm{p} \geq \begin{pmatrix} - \bm{\hat{E}} \bm{a}_{max} - \bm{\hat{D}}_2 \bm{p} \\ - \bm{\hat{E}} \bm{a}_{max} + \bm{\hat{D}}_2 \bm{p} \end{pmatrix}
\end{eqnarray}

%%
\subsubsection*{タイミング上下限制約}
タイミングが初期,終端時刻の間に含まれる制約は次式で表される.
\begin{eqnarray}
  &&t_s \leq t_l \leq t_f \ \ (l = 1,2,\cdots,N_{\mathit{tm}}) \\
  \Leftrightarrow&&
  t_s \bm{1} \leq \bm{t} \leq t_f \bm{1} \label{eq:bspline-timing-min-max-constraint} \\
  \Leftrightarrow&&
  \begin{pmatrix} \bm{I} \\ - \bm{I} \end{pmatrix} \bm{t} \geq \begin{pmatrix} t_s \bm{1} \\ - t_f \bm{1} \end{pmatrix}
\end{eqnarray}
これは,逐次二次計画法の中で,次式の不等式制約となる.
\begin{eqnarray}
  \begin{pmatrix} \bm{I} \\ - \bm{I} \end{pmatrix} \Delta \bm{t} \geq \begin{pmatrix} t_s \bm{1} - \bm{t} \\ - t_f \bm{1} + \bm{t} \end{pmatrix} \\
\end{eqnarray}

また,タイミングの順序が入れ替わることを許容しない場合,その制約は次式で表される.
\begin{eqnarray}
  &&t_l \leq t_{l+1} \ \ (l = 1,2,\cdots,N_{\mathit{tm}}-1) \\
  \Leftrightarrow&&
  - t_l + t_{l+1} \geq 0 \ \ (l = 1,2,\cdots,N_{\mathit{tm}}-1) \\
  \Leftrightarrow&&
  \bm{D}_{\mathit{tm}} \bm{t} \geq \bm{0} \label{eq:bspline-timing-order-constraint}
\end{eqnarray}
ただし,
\begin{eqnarray}
  \bm{D}_{\mathit{tm}} = \begin{pmatrix} -1 & 1 &&&& \bm{O} \\ & -1 & 1 &&& \\ &&&\ddots& \\ \bm{O} &&&& -1 & 1 \end{pmatrix} \in \mathbb{R}^{(N_{\mathit{tm}}-1) \times N_{\mathit{tm}}}
\end{eqnarray}
これは,逐次二次計画法の中で,次式の不等式制約となる.
\begin{eqnarray}
  \bm{D}_{\mathit{tm}} \Delta \bm{t} \geq - \bm{D}_{\mathit{tm}} \bm{t}
\end{eqnarray}

%%
\subsubsection*{関節角微分二乗積分最小化}

関節角微分の二乗積分は次式で得られる.
\begin{eqnarray}
  F_{sqr,k}(\bm{p})
  &=& \int_{t_s}^{t_f} \left\| \bm{\theta}^{(k)}(t) \right\|^2 dt \\
  &=& \int_{t_s}^{t_f} \left\| \bm{B}_{n-k}(t) \bm{\hat{D}}_k \bm{p} \right\|^2 dt \\
  &=& \int_{t_s}^{t_f} \left( \bm{B}_{n-k}(t) \bm{\hat{D}}_k \bm{p} \right)^T \left( \bm{B}_{n-k}(t) \bm{\hat{D}}_k \bm{p} \right) dt \\
  &=& \bm{p}^T \left\{ \int_{t_s}^{t_f} \left( \bm{B}_{n-k}(t) \bm{\hat{D}}_k \right)^T \bm{B}_{n-k}(t) \bm{\hat{D}}_k dt \right\} \bm{p} \\
  &=& \bm{p}^T \bm{H}_k \bm{p} \label{eq:bspline-square-integration}
\end{eqnarray}
ただし,
\begin{eqnarray}
  \bm{H}_k
  &=& \int_{t_s}^{t_f} \left( \bm{B}_{n-k}(t) \bm{\hat{D}}_k \right)^T \bm{B}_{n-k}(t) \bm{\hat{D}}_k dt \\
  \bm{B}_{n-k}(t) \bm{\hat{D}}_k
  &=& \begin{pmatrix} \bm{b}_{n-k}^T(t) && \bm{O} \\ &\ddots& \\ \bm{O} && \bm{b}_{n-k}^T(t) \end{pmatrix} \begin{pmatrix} (\bm{D}^k)^T && \bm{O} \\ &\ddots& \\ \bm{O} && (\bm{D}^k)^T \end{pmatrix} \\
  &=& \begin{pmatrix} \bm{b}_{n-k}^T(t) (\bm{D}^k)^T && \bm{O} \\ &\ddots& \\ \bm{O} && \bm{b}_{n-k}^T(t) (\bm{D}^k)^T \end{pmatrix} \\
  &=& \begin{pmatrix} \left( \bm{D}^k \bm{b}_{n-k}(t) \right)^T && \bm{O} \\ &\ddots& \\ \bm{O} && \left( \bm{D}^k \bm{b}_{n-k}(t) \right)^T \end{pmatrix} \\
  \left( \bm{B}_{n-k}(t) \bm{\hat{D}}_k \right)^T \bm{B}_{n-k}(t)
  &=& \begin{pmatrix} \left( \bm{D}^k \bm{b}_{n-k}(t) \right)^T && \bm{O} \\ &\ddots& \\ \bm{O} && \left( \bm{D}^k \bm{b}_{n-k}(t) \right)^T \end{pmatrix}^T \begin{pmatrix} \left( \bm{D}^k \bm{b}_{n-k}(t) \right)^T && \bm{O} \\ &\ddots& \\ \bm{O} && \left( \bm{D}^k \bm{b}_{n-k}(t) \right)^T \end{pmatrix} \\
  &=& \begin{pmatrix} \left( \bm{D}^k \bm{b}_{n-k}(t) \right) \left( \bm{D}^k \bm{b}_{n-k}(t) \right)^T && \bm{O} \\ &\ddots& \\ \bm{O} && \left( \bm{D}^k \bm{b}_{n-k}(t) \right) \left( \bm{D}^k \bm{b}_{n-k}(t) \right)^T \end{pmatrix}
\end{eqnarray}
これを逐次二次計画問題において,二次形式の正則化項として目的関数に加えることで,滑らかな動作が生成されることが期待される.

%%
\subsubsection*{動作期間の最小化}

動作期間$(t_f - t_s)$の二乗は次式で表される.
\begin{eqnarray}
  F_{\mathit{duration}}(\bm{t})
  &=& \left| t_1 - t_{N_{\mathit{tm}}} \right|^2 \\
  &=& \bm{t}^T \begin{pmatrix} 1 & & -1 \\ & & \\ -1 & & 1 \end{pmatrix} \bm{t} \label{eq:bspline-motion-duration}
\end{eqnarray}
ただし,初期時刻$t_s = t_1$,終端時刻$t_f = t_{N_{\mathit{tm}}}$がタイミングベクトル$\bm{t}$の最初,最後の要素であるとする.
これを逐次二次計画問題において,二次形式の正則化項として目的関数に加えることで,短時間でタスクを実現する動作が生成されることが期待される.
\\
\\

\subsubsection{Bスプラインを用いた関節軌道生成の実装}
\input{bspline-configuration-task}
%%%%%
\subsection{Bスプラインを用いた動的動作の生成} \label{sec:dynamic}
\input{bspline-dynamic-configuration-task}
%%%%%
\subsection{離散的な幾何目標に対する逆運動学計算} \label{sec:discrete-ik}
\subsubsection{離散的な幾何目標に対する逆運動学計算の理論}
%%
\subsubsection*{min/max関数の微分可能関数近似}

minimum/maximum関数
\begin{eqnarray}
  F_{min}(\bm{x}; f_1, \cdots, f_K) &\eqdef& \min (f_1(\bm{x}), \cdots, f_K(\bm{x})) \label{eq:original-min} \\
  F_{max}(\bm{x}; f_1, \cdots, f_K) &\eqdef& \max (f_1(\bm{x}), \cdots, f_K(\bm{x})) \label{eq:original-max}
\end{eqnarray}
を連続かつ微分可能な関数で近似したsmooth minimum/maximum関数として,次式を用いることができる\footnote{\url{https://en.wikipedia.org/wiki/Smooth_maximum}}.
\begin{eqnarray}
  \mathcal{S}_{\alpha}(\bm{x}; f_1, \cdots, f_K) \eqdef \frac{\sum_{k=1}^{K} f_k(\bm{x}) e^{\alpha f_k(\bm{x})}}{\sum_{k=1}^{K} e^{\alpha f_k(\bm{x})}} \label{eq:smooth-max}
\end{eqnarray}
この関数は以下の性質をもつ.
\begin{eqnarray}
  \alpha \to - \inf &のとき& \mathcal{S}_{\alpha} \to F_{min} \\
  \alpha \to \inf &のとき& \mathcal{S}_{\alpha} \to F_{max}
\end{eqnarray}

%%
\subsubsection*{離散的な目標に対するタスク関数の微分可能関数近似}

タスク関数として
$\bm{e}_1(\bm{q}), \cdots, \bm{e}_K(\bm{q}) \in \mathbb{R}^{N_e}$が与えられているときに,
これらのタスク関数のいずれかをゼロにするコンフィギュレーション$\bm{q} \in \mathbb{R}^{N_q}$を求める問題を考える.
複数個の目標位置のいずれかにリーチングする逆運動学問題などがこの問題に含まれる.

この問題は次式で表される.
\begin{eqnarray}
  \bm{e}_k(\bm{q}) = \bm{0} \ \ (kは1,\cdots,Kのいずれか)
\end{eqnarray}
これは次式と同値である.
\begin{eqnarray}
  &&\bm{e}_{min}(\bm{q}) = \bm{0} \\
  &&{\rm where \ \ } \bm{e}_{min}(\bm{q}) \eqdef \argmin_{\bm{e}_k \in \mathcal{E}} \| \bm{e}_k(\bm{q}) \|^2 \in \mathbb{R}^{N_e} \\
  &&\phantom{\rm where \ \ }\mathcal{E} \eqdef \{ \bm{e}_1,\cdots,\bm{e}_K \}
\end{eqnarray}
タスク関数$\bm{e}_{min}(\bm{q})$のヤコビ行列$\frac{\partial \bm{e}_{min}(\bm{q})}{\partial \bm{q}}$が導出できれば,
\chapref{chap:fundamental}の定式化により最適化計算を行うことで
コンフィギュレーション$\bm{q}$を求めることができる.
しかし,$\bm{e}_{min}(\bm{q})$は一般に,最小の$\bm{e}_k$が切り替わる点において微分不可能であり,
ヤコビ行列を求めることができない.

\eqref{eq:smooth-max}では,
$f_k(\bm{x}) \in \mathbb{R} \ (k=1,\cdots,K)$
の
$\dfrac{e^{\alpha f_k(\bm{x})}}{\sum_{k=1}^{K} e^{\alpha f_k(\bm{x})}}$
による重み付けした和をとることで,
min/maxの微分可能関数近似を得ている.
この近似をスカラ値関数からベクトル値関数へと拡張して,
$\bm{e}_{min}(\bm{q})$を次式の微分可能関数で近似する.
\begin{eqnarray}
  &&\bm{\hat{e}}_{min}(\bm{q}) \eqdef
  \dfrac{1}{ \sum_{\bm{e}_k \in \mathcal{E}} \exp(-\alpha \| \bm{e}_k(\bm{q}) \|^2) }
  \sum_{\bm{e}_k \in \mathcal{E}} \exp(-\alpha \| \bm{e}_k(\bm{q}) \|^2) \bm{e}_k(\bm{q}) \in \mathbb{R}^{N_e} \label{eq:smooth-task-func}
\end{eqnarray}
$\alpha$は正の定数で大きいほど近似精度が増す.
タスク関数$\bm{\hat{e}}_{min}(\bm{q})$のヤコビ行列$\frac{\partial \bm{\hat{e}}_{min}(\bm{q})}{\partial \bm{q}}$は,
解析的に導出可能である.

%%
\subsubsection*{contact-invariant-optimizationにおける微分可能関数近似 (参考)}
contact-invariant-optimizationの論文
\footnote{
  Discovery of complex behaviors through contact-invariant optimization,
  I. Mordatch, et. al.,
  ACM Transactions on Graphics 31.4, 43, 2012.
}
の4.1節では,minimum関数を含むタスク関数が以下のように近似されている.
\begin{eqnarray}
  &&\bm{\hat{e}}_{min}(\bm{q}) \eqdef
  \dfrac{1}{ \sum_{\bm{e}_k \in \mathcal{E}} \eta(\bm{e}_k(\bm{q})) }
  \sum_{\bm{e}_k \in \mathcal{E}} \eta(\bm{e}_k(\bm{q})) \bm{e}_k(\bm{q}) \in \mathbb{R}^{N_e} \\
  &&{\rm where \ \ } \eta(\bm{e}_k(\bm{q})) = \frac{1}{1 + \beta \| \bm{e}_k(\bm{q}) \|^2} \in \mathbb{R}
\end{eqnarray}
$\beta$は正の定数で,論文では$10^4$としている.
これは,\eqref{eq:smooth-task-func}における
$\exp(-\alpha \| \bm{e}_k(\bm{q}) \|^2)$を$\eta(\bm{e}_k(\bm{q}))$で置き換えたものである.

%%
\subsubsection*{LogSumExpによる微分可能関数近似 (参考)}

\eqref{eq:original-min},\eqref{eq:original-max}のminimum/maximum関数
を連続かつ微分可能な関数で近似したsmooth minimum/maximum関数として,
LogSumExp関数を用いることができる\footnote{\url{https://en.wikipedia.org/wiki/Smooth_maximum}}.
\begin{eqnarray}
  LSE_{\varepsilon}(\bm{x}; f_1, \cdots, f_K) \eqdef \frac{\log \left( \sum_{k=1}^{K} \exp( \varepsilon f_k(\bm{x})) \right) }{\varepsilon} \label{eq:lse-smooth-max}
\end{eqnarray}
$\varepsilon$が負のときminimum関数,正のときmaximum関数の近似となり,絶対値が大きいほど近似精度が増す.

この関数は,重み付け和の形式ではないため,
\eqref{eq:smooth-task-func}のようにスカラ値関数からベクトル値関数へ拡張することができない.

タスク関数のノルム二乗として表される最適化の目的関数
\begin{eqnarray}
  F(\bm{q}) \eqdef \min_{\bm{e}_k \in \mathcal{E}} \| \bm{e}_k(\bm{q}) \|^2 \in \mathbb{R}
\end{eqnarray}
は,次の$\hat{F}(\bm{q})$として近似できる.
\begin{eqnarray}
  && \hat{F}(\bm{q}) \approx
  \frac{\log\left(\sum_{\bm{e}_k \in \mathcal{E}} \exp(- \varepsilon \| \bm{e}_k(\bm{q}) \|^2)\right)}{- \varepsilon}
\end{eqnarray}
\eqref{eq:lse-smooth-max}の$\varepsilon$を改めて$- \varepsilon$と置き直した.$\varepsilon$が大きいほど近似精度が増す.

近似目的関数$\hat{F}(\bm{q})$の勾配は次式で表される.
\begin{eqnarray}
  \frac{\partial \hat{F}(\bm{q})}{\partial \bm{q}} =
  \frac{\sum_{\bm{e}_k \in \mathcal{E}} 2 \varepsilon \exp(- \varepsilon \| \bm{e}_k(\bm{q}) \|^2) \left(\frac{\partial \bm{e}_k(\bm{q})}{\partial \bm{q}}\right)^T \bm{e}_k(\bm{q})}
       {\varepsilon \sum_{\bm{e}_k \in \mathcal{E}} \exp(- \varepsilon \| \bm{e}_k(\bm{q}) \|^2)}
\end{eqnarray}

近似目的関数$\hat{F}(\bm{q})$のヘッセ行列も解析的に導出可能である.
(タスク関数を考える場合,そのヤコビ行列が求まれば,\chapref{chap:fundamental}のように目的関数のヘッセ行列は導出可能である.しかし,今回のように目的関数を直接扱う場合は,そのヘッセ行列を陽に導出する必要がある.)

\subsubsection{離散的な幾何目標に対する逆運動学計算の実装}
\input{discrete-kinematics-configuration-task}

%%%%%%%%%%%%%%%%%%%%%%
\section{補足} \label{chap:appendix}
%%%%%
\subsection{既存のロボット基礎クラスの拡張} \label{sec:base-extention}
\input{extended-joint-link}
%%%%%
\subsection{環境と接触するロボットの関節・リンク構造} \label{sec:robot-environment}
\input{contact-kinematics}
\input{robot-environment}
%%%%%
\subsection{irteusのinverse-kinematics互換関数} \label{sec:ik-wrapper}
\input{inverse-kinematics-wrapper}
%%%%%
\subsection{関節トルク勾配の計算} \label{sec:torque-jacobian}
\input{torque-gradient}

\end{document}

