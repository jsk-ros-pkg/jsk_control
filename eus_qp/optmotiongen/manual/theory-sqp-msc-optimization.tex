\eqref{eq:ik-opt-3a}の最適化問題に逐次二次計画法などの制約付き非線形最適化手法を適用すると,
初期値から勾配方向に進行して至る局所最適解が得られると考えられる.
したがって解は初期値に強く依存する.

\eqref{eq:ik-opt-3a}の代わりに,以下の最適化問題を考える.
\begin{eqnarray}
  &&\min_{\bm{\hat{q}}} \ \sum_{i \in \mathcal{I}} \left\{ F(\bm{q}^{(i)}) + k_{\mathit{msc}} F_{\mathit{msc}}(\bm{\hat{q}}; i) \right\} \label{eq:solution-candidate-opt} \\
  &&{\rm s.t.} \ \  \bm{A} \bm{q}^{(i)} = \bm{\bar{b}} \ \ \ \ i \in \mathcal{I} \\
  &&\phantom{\rm s.t.} \ \  \bm{C} \bm{q}^{(i)} \geq \bm{\bar{d}} \ \ \ \ i \in \mathcal{I} \\
  &&{\rm where} \ \ \bm{\hat{q}} \eqdef \begin{pmatrix} \bm{q}^{(1)T} & \bm{q}^{(2)T} & \cdots & \bm{q}^{(N_{\mathit{msc}})T} \end{pmatrix}^T \\
  &&\phantom{\rm where} \ \ \mathcal{I} \eqdef \{ 1,2,\cdots,N_{\mathit{msc}} \} \\
  &&\phantom{\rm where} \ \ F_{\mathit{msc}}(\bm{\hat{q}}; i) \eqdef - \frac{1}{2} \sum_{\substack{j \in \mathcal{I} \\ j \not= i}} \log \| \bm{d}(\bm{q}^{(i)}, \bm{q}^{(j)}) \|^2 \\
  &&\phantom{\rm where} \ \ \bm{d}(\bm{q}^{(i)}, \bm{q}^{(j)}) \eqdef \bm{q}^{(i)} - \bm{q}^{(j)}
\end{eqnarray}
$N_{\mathit{msc}}$は解候補の個数で,事前に与えるものとする.$\mathit{msc}$は複数解候補(multiple solution candidates)を表す.
これは,複数の解候補を同時に探索し,それぞれの解候補$\bm{q}^{(i)}$が本来の目的関数$F(\bm{q}^{(i)})$を小さくして,なおかつ,解候補どうしの距離が大きくなるように最適化することを表している.
これにより,初期値に依存した唯一の局所解だけでなく,そこから離れた複数の局所解を得ることが可能となり,通常の最適化に比べてより良い解が得られることが期待される.
以降では,解候補どうしの距離のコストを表す項$F_{\mathit{msc}}(\bm{\hat{q}}; i)$を解候補分散項と呼ぶ
\footnote{解分散項の$\log$を無くすことは適切ではない.なぜなら,$d = \| \bm{d}(\bm{q}^{(i)}, \bm{q}^{(j)}) \|$として,解分散項の勾配は,
\begin{eqnarray}
  \frac{\partial}{\partial d}\left(- \frac{1}{2} \log d^2 \right) = - \frac{1}{d} \to - \infty \ \ (d \to +0) \hspace{10mm}
  \frac{\partial}{\partial d}\left(- \frac{1}{2} \log d^2 \right) = - \frac{1}{d} \to 0 \ \ (d \to \infty)
\end{eqnarray}
となり,最適化により,コンフィギュレーションが近いときほど離れるように更新し,遠くなるとその影響が小さくなる効果が期待される.それに対し,$\log$がない場合の勾配は,
\begin{eqnarray}
  \frac{\partial}{\partial d}\left(- \frac{1}{2} d^2 \right) = - d \to 0 \ \ (d \to +0) \hspace{10mm}
  \frac{\partial}{\partial d}\left(- \frac{1}{2} d^2 \right) = - d \to - \infty \ \ (d \to \infty)
\end{eqnarray}
となり,コンフィギュレーションが遠くなるほど離れるように更新し,近いときはその影響が小さくなる.これは,コンフィギュレーションが一致する勾配ゼロの点と,無限に離れ発散する最適値をもち,これらは最適化において望まない挙動をもたらす.
}.

解候補分散項のヤコビ行列,ヘッセ行列の各成分は次式で得られる\footnote{ヘッセ行列の導出は以下を参考にした.\url{https://math.stackexchange.com/questions/175263/gradient-and-hessian-of-general-2-norm}}.
\begin{subequations}
\begin{eqnarray}
  \nabla_i F_{\mathit{msc}}(\bm{\hat{q}}; i) &=& \frac{\partial F_{\mathit{msc}}(\bm{\hat{q}}; i)}{\partial \bm{q}^{(i)}} \\
  &=& - \frac{1}{2} \sum_{\substack{j \in \mathcal{I} \\ j \not= i}} \frac{\partial}{\partial \bm{q}^{(i)}} \log \| \bm{d}(\bm{q}^{(i)}, \bm{q}^{(j)}) \|^2 \\
  &=& - \sum_{\substack{j \in \mathcal{I} \\ j \not= i}} \frac{1}{\| \bm{d}(\bm{q}^{(i)}, \bm{q}^{(j)}) \|^2} \left( \frac{\partial \bm{d}(\bm{q}^{(i)}, \bm{q}^{(j)})}{\partial \bm{q}^{(i)}} \right)^T \bm{d}(\bm{q}^{(i)}, \bm{q}^{(j)})\\
  &=& - \sum_{\substack{j \in \mathcal{I} \\ j \not= i}} \frac{\bm{d}(\bm{q}^{(i)}, \bm{q}^{(j)})}{\| \bm{d}(\bm{q}^{(i)}, \bm{q}^{(j)}) \|^2} \\
\end{eqnarray}
\end{subequations}
\begin{subequations}
\begin{eqnarray}
  \nabla_k F_{\mathit{msc}}(\bm{\hat{q}}; i) &=& \frac{\partial F_{\mathit{msc}}(\bm{\hat{q}}; i)}{\partial \bm{q}^{(k)}} \ \ \ \ k \in \mathcal{I} \land k \not= i \\
  &=& - \frac{1}{2} \sum_{\substack{j \in \mathcal{I} \\ j \not= i}} \frac{\partial}{\partial \bm{q}^{(k)}} \log \| \bm{d}(\bm{q}^{(i)}, \bm{q}^{(j)}) \|^2 \\
  &=& - \frac{1}{2} \frac{\partial}{\partial \bm{q}^{(k)}} \log \| \bm{d}(\bm{q}^{(i)}, \bm{q}^{(k)}) \|^2 \\
  &=& - \frac{1}{\| \bm{d}(\bm{q}^{(i)}, \bm{q}^{(k)}) \|^2} \left( \frac{\partial \bm{d}(\bm{q}^{(i)}, \bm{q}^{(k)})}{\partial \bm{q}^{(k)}} \right)^T \bm{d}(\bm{q}^{(i)}, \bm{q}^{(k)})\\
  &=& \frac{\bm{d}(\bm{q}^{(i)}, \bm{q}^{(k)})}{\| \bm{d}(\bm{q}^{(i)}, \bm{q}^{(k)}) \|^2} \\
\end{eqnarray}
\end{subequations}
\begin{subequations}
\begin{eqnarray}
  \nabla^2_{ii} F_{\mathit{msc}}(\bm{\hat{q}}; i) &=& \frac{\partial^2 F_{\mathit{msc}}(\bm{\hat{q}}; i)}{\partial \bm{q}^{(i) 2}} \\
  &=& - \sum_{\substack{j \in \mathcal{I} \\ j \not= i}} \frac{\partial}{\partial \bm{q}^{(i)}} \left( \left\{ \| \bm{d}(\bm{q}^{(i)}, \bm{q}^{(j)}) \|^2 \right\}^{-1} \bm{d}(\bm{q}^{(i)}, \bm{q}^{(j)}) \right) \\
  &=& - \sum_{\substack{j \in \mathcal{I} \\ j \not= i}} \left( - 2 \left\{ \| \bm{d}(\bm{q}^{(i)}, \bm{q}^{(j)}) \|^2 \right\}^{-2} \bm{d}(\bm{q}^{(i)}, \bm{q}^{(j)}) \bm{d}(\bm{q}^{(i)}, \bm{q}^{(j)})^T + \left\{ \| \bm{d}(\bm{q}^{(i)}, \bm{q}^{(j)}) \|^2 \right\}^{-1} \bm{I} \right) \\
  &=& - \sum_{\substack{j \in \mathcal{I} \\ j \not= i}} \left( - \frac{2}{\| \bm{d}(\bm{q}^{(i)}, \bm{q}^{(j)}) \|^4} \bm{d}(\bm{q}^{(i)}, \bm{q}^{(j)}) \bm{d}(\bm{q}^{(i)}, \bm{q}^{(j)})^T + \frac{1}{ \| \bm{d}(\bm{q}^{(i)}, \bm{q}^{(j)}) \|^2 } \bm{I} \right) \\
  &=& - \sum_{\substack{j \in \mathcal{I} \\ j \not= i}} \bm{H}(\bm{q}^{(i)}, \bm{q}^{(j)})
\end{eqnarray}
\end{subequations}
ただし,
\begin{eqnarray}
  \bm{H}(\bm{q}^{(i)}, \bm{q}^{(j)}) \eqdef
  - \frac{2}{\| \bm{d}(\bm{q}^{(i)}, \bm{q}^{(j)}) \|^4} \bm{d}(\bm{q}^{(i)}, \bm{q}^{(j)}) \bm{d}(\bm{q}^{(i)}, \bm{q}^{(j)})^T + \frac{1}{ \| \bm{d}(\bm{q}^{(i)}, \bm{q}^{(j)}) \|^2 } \bm{I}
\end{eqnarray}
\begin{subequations}
\begin{eqnarray}
  \nabla^2_{ik} F_{\mathit{msc}}(\bm{\hat{q}}; i) &=& \frac{\partial^2 F_{\mathit{msc}}(\bm{\hat{q}}; i)}{\partial \bm{q}^{(i)} \partial \bm{q}^{(k)}} \ \ \ \ k \in \mathcal{I} \land k \not= i \\
  &=& - \sum_{\substack{j \in \mathcal{I} \\ j \not= i}} \frac{\partial}{\partial \bm{q}^{(k)}} \left( \left\{ \| \bm{d}(\bm{q}^{(i)}, \bm{q}^{(j)}) \|^2 \right\}^{-1} \bm{d}(\bm{q}^{(i)}, \bm{q}^{(j)}) \right) \\
  &=& - \frac{\partial}{\partial \bm{q}^{(k)}} \left( \left\{ \| \bm{d}(\bm{q}^{(i)}, \bm{q}^{(k)}) \|^2 \right\}^{-1} \bm{d}(\bm{q}^{(i)}, \bm{q}^{(k)}) \right) \\
  &=& - \left( 2 \left\{ \| \bm{d}(\bm{q}^{(i)}, \bm{q}^{(k)}) \|^2 \right\}^{-2} \bm{d}(\bm{q}^{(i)}, \bm{q}^{(k)}) \bm{d}(\bm{q}^{(i)}, \bm{q}^{(k)})^T - \left\{ \| \bm{d}(\bm{q}^{(i)}, \bm{q}^{(k)}) \|^2 \right\}^{-1} \bm{I} \right) \\
  &=& - \frac{2}{\| \bm{d}(\bm{q}^{(i)}, \bm{q}^{(k)}) \|^4} \bm{d}(\bm{q}^{(i)}, \bm{q}^{(k)}) \bm{d}(\bm{q}^{(i)}, \bm{q}^{(k)})^T + \frac{1}{ \| \bm{d}(\bm{q}^{(i)}, \bm{q}^{(k)}) \|^2 } \bm{I} \\
  &=& \bm{H}(\bm{q}^{(i)}, \bm{q}^{(k)})
\end{eqnarray}
\end{subequations}
\begin{subequations}
\begin{eqnarray}
  \nabla^2_{kk} F_{\mathit{msc}}(\bm{\hat{q}}; i) &=& \frac{\partial^2 F_{\mathit{msc}}(\bm{\hat{q}}; i)}{\partial \bm{q}^{(k)2}} \ \ \ \ k \in \mathcal{I} \land k \not= i\\
  &=& \frac{\partial}{\partial \bm{q}^{(k)}} \left( \left\{ \| \bm{d}(\bm{q}^{(i)}, \bm{q}^{(k)}) \|^2 \right\}^{-1} \bm{d}(\bm{q}^{(i)}, \bm{q}^{(k)}) \right) \\
  &=& - \left( - \frac{2}{\| \bm{d}(\bm{q}^{(i)}, \bm{q}^{(k)}) \|^4} \bm{d}(\bm{q}^{(i)}, \bm{q}^{(k)}) \bm{d}(\bm{q}^{(i)}, \bm{q}^{(k)})^T + \frac{1}{ \| \bm{d}(\bm{q}^{(i)}, \bm{q}^{(k)}) \|^2 } \bm{I} \right) \\
  &=& - \bm{H}(\bm{q}^{(i)}, \bm{q}^{(k)})
\end{eqnarray}
\end{subequations}
\begin{subequations}
\begin{eqnarray}
  \nabla^2_{kl} F_{\mathit{msc}}(\bm{\hat{q}}; i) &=& \frac{\partial^2 F_{\mathit{msc}}(\bm{\hat{q}}; i)}{\partial \bm{q}^{(k)} \partial \bm{q}^{(l)}} \ \ \ \ k \in \mathcal{I} \land l \in \mathcal{I} \land k \not= i \land l \not= i \land k \not= l\\
  &=& \frac{\partial}{\partial \bm{q}^{(l)}} \left( \left\{ \| \bm{d}(\bm{q}^{(i)}, \bm{q}^{(k)}) \|^2 \right\}^{-1} \bm{d}(\bm{q}^{(i)}, \bm{q}^{(k)}) \right) \\
  &=& \bm{O}
\end{eqnarray}
\end{subequations}
したがって,
解候補分散項のヤコビ行列,ヘッセ行列は次式で表される.
\begin{subequations}
\begin{eqnarray}
  \nabla F_{\mathit{msc}}(\bm{\hat{q}}; i) &=& \frac{\partial F_{\mathit{msc}}(\bm{\hat{q}}; i)}{\partial \bm{\hat{q}}} \\
  &=&
  \begin{pmatrix}
  \frac{\bm{d}(\bm{q}^{(i)}, \bm{q}^{(1)})}{\| \bm{d}(\bm{q}^{(i)}, \bm{q}^{(1)}) \|^2} \\
  \vdots \\
  \frac{\bm{d}(\bm{q}^{(i)}, \bm{q}^{(i-1)})}{\| \bm{d}(\bm{q}^{(i)}, \bm{q}^{(i-1)}) \|^2} \\
  - \sum_{\substack{j \in \mathcal{I} \\ j \not= i}} \frac{\bm{d}(\bm{q}^{(i)}, \bm{q}^{(j)})}{\| \bm{d}(\bm{q}^{(i)}, \bm{q}^{(j)}) \|^2} \\
  \frac{\bm{d}(\bm{q}^{(i)}, \bm{q}^{(i+1)})}{\| \bm{d}(\bm{q}^{(i)}, \bm{q}^{(i+1)}) \|^2} \\
  \vdots \\
  \frac{\bm{d}(\bm{q}^{(i)}, \bm{q}^{(N_{\mathit{msc}})})}{\| \bm{d}(\bm{q}^{(i)}, \bm{q}^{(N_{\mathit{msc}})}) \|^2}
  \end{pmatrix} \\
  \bm{v}_{\mathit{msc}} &\eqdef&
  \sum_{i \in \mathcal{I}} \nabla F_{\mathit{msc}}(\bm{\hat{q}}; i) \\
  &=&
  2
  \begin{pmatrix}
  - \sum_{\substack{j \in \mathcal{I} \\ j \not= 1}} \frac{\bm{d}(\bm{q}^{(1)}, \bm{q}^{(j)})}{\| \bm{d}(\bm{q}^{(1)}, \bm{q}^{(j)}) \|^2} \\
  \vdots \\
  - \sum_{\substack{j \in \mathcal{I} \\ j \not= N_{\mathit{msc}}}} \frac{\bm{d}(\bm{q}^{(N_{\mathit{msc}})}, \bm{q}^{(j)})}{\| \bm{d}(\bm{q}^{(N_{\mathit{msc}})}, \bm{q}^{(j)}) \|^2} \\
  \end{pmatrix} \label{eq:sqp-msc-dispersion-matrix}
\end{eqnarray}
\end{subequations}
\begin{subequations}
\begin{eqnarray}
  \nabla^2 F_{\mathit{msc}}(\bm{\hat{q}}; i) &=& \frac{\partial^2 F_{\mathit{msc}}(\bm{\hat{q}}; i)}{\partial \bm{\hat{q}}^2} \\
  &=&
  \bordermatrix{
    & 1 & \cdots & i-1 & i & i+1 & \cdots & N_{\mathit{msc}} \cr
    1 & - \bm{H}_{i,1} & & & \bm{H}_{i,1} & & &\cr
    \vdots & & \ddots & & \vdots & & & \cr
    i-1 & & & - \bm{H}_{i,i-1} & \bm{H}_{i,i-1} & & & \cr
    i & \bm{H}_{i,1} & \cdots & \bm{H}_{i,i-1} & - \sum_{\substack{j \in \mathcal{I} \\ j \not= i}} \bm{H}_{i,j} & \bm{H}_{i,i+1} & \cdots & \bm{H}_{i,N_{\mathit{msc}}} \cr
    i+1 & & & & \bm{H}_{i,i+1} & - \bm{H}_{i,i+1} & & \cr
    \vdots & & & & \vdots & & \ddots & \cr
    N_{\mathit{msc}} & & & & \bm{H}_{i,N_{\mathit{msc}}} & & & - \bm{H}_{i, N_{\mathit{msc}}} \cr
  } \\
  \bm{W}_{\mathit{msc}} &\eqdef&
  \sum_{i \in \mathcal{I}} \nabla^2 F_{\mathit{msc}}(\bm{\hat{q}}; i) \\
  &=&
  2
  \begin{pmatrix}
    - \sum_{\substack{j \in \mathcal{I} \\ j \not= i}} \bm{H}_{1,j} & \bm{H}_{1,2} & \cdots & \bm{H}_{1,N_{\mathit{msc}}} \\
    \bm{H}_{2,1} & - \sum_{\substack{j \in \mathcal{I} \\ j \not= i}} \bm{H}_{2,j} & & \bm{H}_{2,N_{\mathit{msc}}} \\
    \vdots & & \ddots & \vdots \\
    \bm{H}_{N_{\mathit{msc}},1} & \bm{H}_{N_{\mathit{msc}},2} & \cdots & - \sum_{\substack{j \in \mathcal{I} \\ j \not= i}} \bm{H}_{N_{\mathit{msc}},j}
  \end{pmatrix} \label{eq:sqp-msc-dispersion-vector}
\end{eqnarray}
\end{subequations}
ただし,$\bm{H}(\bm{q}^{(i)}, \bm{q}^{(j)})$を$\bm{H}_{i,j}$と略して記す.
また,$\bm{d}(\bm{q}^{(i)}, \bm{q}^{(j)}) = - \bm{d}(\bm{q}^{(j)}, \bm{q}^{(i)}), \ \bm{H}_{i,j} = \bm{H}_{j,i}$を利用した.

解候補分散項$\sum_{i \in \mathcal{I}} F_{\mathit{msc}}(\bm{\hat{q}}; i)$による二次計画問題の目的関数(\eqref{eq:sqp-1a})は次式で表される.
\begin{eqnarray}
  &&\sum_{i \in \mathcal{I}} \left\{ F_{\mathit{msc}}(\bm{\hat{q}}_k; i) + \nabla F_{\mathit{msc}}(\bm{\hat{q}}_k; i)^T \Delta \bm{\hat{q}}_k + \frac{1}{2} \Delta \bm{\hat{q}}_k^T \nabla^2 F_{\mathit{msc}}(\bm{\hat{q}}_k; i) \Delta \bm{\hat{q}}_k \right\} \\
  &=&
  \sum_{i \in \mathcal{I}} F_{\mathit{msc}}(\bm{\hat{q}}_k; i)
  + \left\{ \sum_{i \in \mathcal{I}} \nabla F_{\mathit{msc}}(\bm{\hat{q}}_k; i) \right\}^T \Delta \bm{\hat{q}}_k
  + \frac{1}{2} \Delta \bm{\hat{q}}_k^T \left\{ \sum_{i \in \mathcal{I}} \nabla^2 F_{\mathit{msc}}(\bm{\hat{q}}_k; i) \right\} \Delta \bm{\hat{q}}_k \\
  &=&
  \sum_{i \in \mathcal{I}} F_{\mathit{msc}}(\bm{\hat{q}}_k; i) + \bm{v}_{\mathit{msc}}^T \Delta \bm{\hat{q}}_k + \frac{1}{2} \Delta \bm{\hat{q}}_k^T \bm{W}_{\mathit{msc}} \Delta \bm{\hat{q}}_k
\end{eqnarray}

$\bm{W}_{\mathit{msc}}$が必ずしも半正定値行列ではないことに注意する必要がある.
以下のようにして$\bm{W}_{\mathit{msc}}$に近い正定値行列を計算し用いることで対処する
\footnote{$\bm{W}_{\mathit{msc}}$が対称行列であることから,以下を参考にした.\url{https://math.stackexchange.com/questions/648809/how-to-find-closest-positive-definite-matrix-of-non-symmetric-matrix\#comment1689831_649522}}.
$\bm{W}_{\mathit{msc}}$が次式のように固有値分解されるとする.
\begin{eqnarray}
  \bm{W}_{\mathit{msc}} = \bm{V}_{\mathit{msc}} \bm{D}_{\mathit{msc}} \bm{V}_{\mathit{msc}}^{-1}
\end{eqnarray}
ただし,$\bm{D}_{\mathit{msc}}$は固有値を対角成分にもつ対角行列,$\bm{V}_{\mathit{msc}}$は固有ベクトルを並べた行列である.
このとき$\bm{W}_{\mathit{msc}}$に近い正定値行列$\bm{\tilde{W}}_{\mathit{msc}}$は次式で得られる.
\begin{eqnarray}
  \bm{\tilde{W}}_{\mathit{msc}} = \bm{V}_{\mathit{msc}} \bm{D}_{\mathit{msc}}^+ \bm{V}_{\mathit{msc}}^{-1}
\end{eqnarray}
ただし,$\bm{D}_{\mathit{msc}}^+$は$\bm{D}_{\mathit{msc}}$の対角成分のうち,負のものを$0$で置き換えた対角行列である.

\eqref{eq:solution-candidate-opt}において,
解候補を分散させながら,最終的に本来の目的関数を最小にする解を得るために,
SQPのイテレーションごとに,解候補分散項のスケール$k_{\mathit{msc}}$を次式のように更新することが有効である.
\begin{eqnarray}
  k_{\mathit{msc}} \gets \min ( \gamma_{\mathit{msc}} k_{\mathit{msc}}, k_{\mathit{msc\mathchar`-min}})
\end{eqnarray}
$\gamma_{\mathit{msc}}$は$0< \gamma_{\mathit{msc}} < 1$なるスケール減少率,
$k_{\mathit{msc\mathchar`-min}}$はスケール最小値を表す.
